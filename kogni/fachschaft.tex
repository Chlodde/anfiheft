\subsection{Wer ist die Fachschaft?}
Die Fachschaft Kognitionswissenschaft besteht vollständig aus Studierenden, die sich freiwillig treffen, um das Leben und Studieren in Tübingen im Sinne der Studierenden zu verbessern. Da das Studienfach Kognitionswissenschaft inhaltlich und organisatiorisch viele Überschneidungen mit der Informatik besitzt, arbeiten wir stark mit der Fachschaft Informatik zusammen.
Wir planen Feste und Ausflüge, beschäftigen Mentoren und Tutoren und sind die Schnittstelle zur Organisation der Uni und den Dozenten.

\subsection{Was macht die Fachschaft?}
Die Fachschaft übernimmt zahlreiche Aufgaben im Hochschulalltag. Die wichtigsten wollen wir hier kurz nennen. Wenn euch einige dieser Gebiete interessieren, kommt doch mal vorbei und macht mit! Aktuell finden die Fachschaftstreffen wöchentlich online statt. Ein Link zur Teilnahme wird wöchentlich über den kogni-fachschaft-Verteiler geschickt.

\textbf{Erstsemesterbetreuung}

Jedes Semester strömen ca. 60 neue Kogni-Erstsemester an die Uni und haben erstmal keine Ahnung, wie es dort so läuft. Wie ihr vielleicht schon gemerkt habt, versuchen wir euch den Einstieg durch unsere Veranstaltungen, den Brief und dieses Heft einfacher zu machen. Dafür setzten sich Semester für Semester wieder einige Fachschaftler zusammen und überarbeiten unsere Materialien für euch, planen Veranstaltungen und überlegen, was man noch so machen könnte. Außerdem kümmern sich die ErstsemestermentorInnen um die typischen Fragen zu Beginn des Studiums.\\
Falls ihr Lust habt, dafür zu sorgen, dass kommende Erstsemester auch ein (hoffentlich) so cooles Programm, oder ein noch besseres bekommen, dann seid ihr hier genau richtig!\\

\textbf{Gremien}

Wir entsenden jährlich Vertreter in die Studienkommission und den Prüfungsausschuss, um aktiv an der Hochschulpolitik teilzunehmen und die Interessen der Studierenden zu vertreten. Ein Übersicht über die Gremien und unsere derzeitigen Vertreter findet ihr auf der Website der Fachschaft Kognitionswissenschaft unter dem Reiter Gremien\footnote{\url{https://www.fs-kogni.uni-tuebingen.de/fachschaft/ueber_uns/}}. \\	%TODO insert \link{}{}?

\pagebreak

\textbf{Partys und Feste}
\\ \corona \\
So gut wie jedes Semester gibt es eine große Party im Clubhaus Tübingen. Zusammen mit der Fachschaft Informatik und der Fachschaft Psychologie veranstalten wir einmal im Semester ein Clubhausfest, auf das alle Studierende aller Fachrichtungen eingeladen sind. Außerdem gibt es eine Weihnachtsfeier und ein Sommerfest mit der Fachschaft Informatik.

\textbf{Kommunikation}

E-Mails sind im Zeitalter der vielen Social-Media-Kanäle ein stabiles, unabhängiges und flexibles Kommunikationsmedium. Daher betreiben wir eine Mailingliste zum Austausch für alle Studierende der Kognitionswissenschaft. Wie ihr euch anmelden könnt steht weiter unten. \\

\subsection{Wie kann ich mitmachen?}
\coronabox{Aufgrund der Corona-Pandemie und den einhergehenden Hygienevorschriften findet die Fachschaftssitzung wöchentlich online per Zoom-Meeting statt. \\ Den Link zur Teilnahme findet ihr auf der Fachschaft-Homepage oben rechts im Kalender \\ \url{https://www.fs-kogni.uni-tuebingen.de/}.\\
 Wir werden in der ersten Sitzung diskutieren, ob eine Präsenz-Fachschaftssitzung möglich ist. Wenn ja wird zur Sicherheit aller die Sitzung im Hybridformat gehalten.}
Wir sind immer auf der Suche nach motivierten Menschen! Mitmachen kann bei uns jeder, der Lust hat sich für seine Kommilitonen zu engagieren. Wie ihr nun gesehen habt, kann man sich bei uns in den verschiedensten Bereichen engagieren. Wenn euch ein, zwei oder mehrere der Bereiche interessieren, dann kommt doch einfach mal vorbei. Bei uns braucht man kein Vorwissen, alles was ihr wissen müsst, erklären wir euch gerne!
Wann und wo wir uns treffen erfahrt ihr von einer der Erstsemester-MentorInnen oder indem ihr uns eine E-Mail schreibt \footnote{\email{kogni-fachschaft@fsi.uni-tuebingen.de}}.

\vfill
\subsection{Andere Fachschaften}
Hin und wieder kann es sinnvoll und notwendig sein auch eine andere Fachschaft zu kontaktieren, wenn es beispielsweise um Veranstaltungen eines anderen Fachbereichs geht. Diese beantworten eure Fragen ebenfalls gerne. Die wichtigsten Kontakte haben wir euch hier aufgelistet:

\textbf{Fachschaft Informatik}\\
E-Mail: \email{fsi@fsi.uni-tuebingen.de}\\
\\
\textbf{Fachschaft Psychologie}\\
E-Mail: \email{info@fs-psycho.uni-tuebingen.de}\\
\\
\textbf{Fachschaft Mathematik}\\
E-Mail: \email{fachschaft@math.uni-tuebingen.de}\\
\\
\textbf{Fachschaft Biologie}\\
E-Mail: \email{fsbio@uni-tuebingen.de}\\
\\
\textbf{Fachschaft Chemie}\\
E-Mail: \email{fschemie@uni-tuebingen.de}\\
\\
\textbf{Fachschaft Physik}\\
E-Mail: \email{fsphysik@uni-tuebingen.de}\\
\\
\textbf{Fachschaft Medizin}\\
E-Mail: \email{ info@fachschaftmedizin.de}\\
\\
Eine vollständige Liste aller Fachschaften findet ihr auf der Website der Uni\footnote{\url{https://uni-tuebingen.de/studium/rund-ums-studium/studentisches-engagement/fachschaften/}}	%TODO insert \link{}{}?

\vfill
