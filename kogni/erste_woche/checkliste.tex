Da in der ersten Woche immer viel los ist, haben wir hier eine kleine Checkliste mit Dingen geschrieben, die ihr auf gar keinen Fall vergessen haben solltet:
\begin{enumerate}[label=$\bigcirc$]
	\item  \textbf{Erstwohnsitz umgemeldet} \\
	Hierzu müsst ihr zum Bürgeramt und euren Mietvertrag und eine Wohnungsgeberbescheinigung (bekommt ihr von eurem Vermieter) mitbringen. Die Ummeldung dauert für gewöhnlich nur wenige Minuten, normalerweise hat man jedoch etwas Wartezeit.
	
	\item \textbf{Rundfunkbeitrag angemeldet}\\
	Diesen Brief bekommt ihr normalerweise, sobald ihr euch nach Tübingen umgemeldet habt. Den Rundfunkbeitrag muss aber nur eine Person pro Haushalt bezahlen. Fragt deshalb am besten eure Mitbewohner, ob diesen schon jemand bezahlt.	
	
	\item  \textbf{Wohnadresse in Campus aktualisiert}\\
	Da die Uni euch hin und wieder wichtige Post schickt, benötigt sie eure aktuelle Adresse. Diese könnt ihr über das CAMPUS-System ändern.
	
	\item \textbf{WLAN-Zugang mit Eduroam eingerichtet}\\
	Mit deinen ZDV-Login-Daten könnt ihr europaweit an allen teilnehmenden Universitäten ins Internet.
	
	\item  \textbf{Neurobiologie und Sinnesphysiologie Seminar gewählt}\\
	Hierzu geht für gewöhnlich in der ersten Vorlesung eine Liste rum, in die man sich ganz altmodisch per Stift eintragen muss. Sollte man in der ersten Vorlesung nicht anwesend sein können, kann man Herrn Mallot auch eine Mail schicken.
	
	\item  \textbf{Zum Übungsbetrieb angemeldet}\\
	Bei Informatik 1 und Mathe 1 muss man, um zur Klausur zugelassen zu werden, eine bestimmte Anzahl an Punkten in den Übungsblättern erreichen. Damit ihr die Übungen
machen könnt, müsst ihr euch in der Veranstaltung dazu anmelden. Wie das geht,
erfahrt ihr in den ersten Vorlesungen (oft sogar direkt in der ersten Vorlesung, also
nicht verpassen!).

	\item \textbf{Auf Versuchspersonen-Listen anmelden}\\
	In den ersten drei Semestern müssen alle Kognis 30 Stunden als Versuchsperson für psychologische Experimente zur Verfügung stehen (Mehr dazu im Abschnitt \textit{Die Versuchspersonenstunden}). Zu den Experimenten anmelden könnt ihr euch unter der Versuche-Mailingliste (Siehe Abschnitt \textit{Mailinglisten}) oder in der Facebook-Gruppe \textit{Versuchspersonenbörse Tübingen}.
	
	
\end{enumerate}