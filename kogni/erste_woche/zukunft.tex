Nachdem ihr nun hoffentlich einen guten Blick über euer erstes Semester bekommen habt, möchten wir euch noch einen kleinen Ausblick auf künftige Semester geben. Welche Veranstaltungen ihr belegen müsst, um euren Bachelorabschluss zu bekommen, regeln Prüfungsordnung und Modulhandbuch\footnote{\url{https://uni-tuebingen.de/fakultaeten/mathematisch-naturwissenschaftliche-fakultaet/fachbereiche/informatik/studium/studiengaenge/kognitionswissenschaft/posmhbsdownloadsinfos/}}. Für euch gilt die Prüfungsordnung von 2017, d.h. der allgemeine Teil von 2015 in Kombination mit der Änderungssatzung 2017 sind relevant. Einen groben Überblick bekommt man aber schon durch den auf der nächsten Seite dargestellten Studienverlaufsplan.\\
\\
Die meisten Veranstaltungen in unserem Studium sind fest vorgegeben. Erst in den späteren Semestern haben wir vier Wahlpflichtmodule: Philosophie, Linguistik, Informatik und Kognition. In diesen Modulen könnt ihr aus den im ALMA\footnote{\url{https://alma.uni-tuebingen.de}} unter dem jeweiligen Modul aufgeführten Veranstaltungen frei wählen. Die Module, die ohne Mehraufwand anrechenbar sind, sind im Reiter Mein Studium -> Studienplaner mit Modulplan.\\
Außerdem habt ihr noch Wahlmöglichkeiten in den Modulen Experimentelle Kognitionswissenschaft und dem Teamprojekt. In experimenteller Kognitionswissenschaft werdet ihr die in den ersten Semestern erworbenen Grundlagen der empirischen Forschung kombiniert mit eurem dazu gewonnenen Wissen zum ersten Mal anwenden und in Kleingruppen jeweils ein Experiment durchführen. Wem dies Spaß macht kann im Teamprojekt wieder empirisch arbeiten und zum Beispiel auch neuere Methoden wie das Elektroenzephalogramm (EEG) verwenden. Wer es lieber technischer mag, kann auch ein eher informatiknahes Projekt mit Schwerpunkt auf künstliche Intelligenz oder Neuronale Netze wählen.

Neben diesen fachbezogenen Wahlbereichen gibt es noch den Bereich Studium Professionale, in dem ihr überfachliche Qualifikationen erlernen sollt. Da das Studium der Kognitionswissenschaft bereits sehr überfachlich ist, können hier alle Veranstaltungen der Uni außer Sportveranstaltungen angerechnet werden.

\newcolumntype{C}[1]{>{\centering\arraybackslash}p{#1}}

\begin{tabular}{| C{0.14\textwidth} | C{0.14\textwidth} | C{0.14\textwidth} | C{0.14\textwidth} | C{0.14\textwidth} | C{0.14\textwidth} |} \hline
 1. Semester & 2. Semester & 3. Semester & 4. Semester & 5. Semester & 6. Semester\\ \hline \hline
 & & & & & \\
 & & & & \scriptsize{Computational} & \\
 & & & & \scriptsize{Neuroscience} & \\
 \scriptsize{Mathematik I} & \scriptsize{Mathematik II} & \scriptsize{Mathematik III} & \scriptsize{WPF Teamprojekt} &  \scriptsize{\textcolor{gray}{6 LP}} & \\
 \scriptsize{\textcolor{gray}{9 LP}} & \scriptsize{\textcolor{gray}{9 LP}} & \scriptsize{\textcolor{gray}{9 LP}} & \scriptsize{\textcolor{gray}{9 LP}} & &  \\
 & & & & & \scriptsize{Bachelorarbeit} \\ \cline{5-5}
 & & & & & \scriptsize{inkl. Vortrag} \\
 & & & & \scriptsize{Kognitions-} & \scriptsize{\textcolor{gray}{15 LP}} \\
 & & & & \scriptsize{informatik} & \\ \cline{1-4}
 
 & & & & \scriptsize{\textcolor{gray}{6 LP}} & \\
 & & & & & \\
 & & & \scriptsize{Philosophie} & & \\ \cline{5-5}
 \scriptsize{Prakt. Informatik I} & \scriptsize{Prakt. Informatik II} & \scriptsize{Algorithmen} & \scriptsize{\textcolor{gray}{6 LP}} & & \\
 \scriptsize{\textcolor{gray}{9 LP}} & \scriptsize{\textcolor{gray}{9 LP}} & \scriptsize{\textcolor{gray}{9 LP}} & & & \\
 & & & & & \\ \cline{4-4}\cline{6-6}
 & & & \cellcolor{lightgray} & & \\
 & & & \cellcolor{lightgray} & \scriptsize{Vertiefung} & \scriptsize{Studium} \\
 & & & \cellcolor{lightgray} & \scriptsize{Kognitions-} & \scriptsize{Professionale} \\ \cline{1-4}
 
 \scriptsize{Mathemath.} & \scriptsize{Mathemath.} & & & \scriptsize{wissenschaft} & \scriptsize{(überfachl.} \\
 \scriptsize{Statistik I} & \scriptsize{Statistik II} & \scriptsize{Linguistik für} & \scriptsize{Language \&} & \scriptsize{\textcolor{gray}{12 LP}}  & \scriptsize{Kompetenzen,} \\
 \scriptsize{\textcolor{gray}{3 LP}} & \scriptsize{\textcolor{gray}{3 LP}} & \scriptsize{Kognitions-} & \scriptsize{Cognition} & & \scriptsize{übK)} \\ \cline{1-2} 
 
 \scriptsize{Computergest.} & \scriptsize{Computergest.} & \scriptsize{wissenschaftler} & \scriptsize{\textcolor{gray}{6 LP}} & & \scriptsize{\textcolor{gray}{9 LP}} \\
 \scriptsize{Statistik I} & \scriptsize{Statistik II} & \scriptsize{\textcolor{gray}{6 LP}} & & & \\
 \scriptsize{\textcolor{gray}{3 LP}} & \scriptsize{\textcolor{gray}{3 LP}} & & & & \\ \cline{1-6}
 
 & \scriptsize{Allgemeine} & &  & \scriptsize{Forschungs-} & \scriptsize{Forschungs-} \\
 \scriptsize{Konzept. u.} & \scriptsize{Psychologie C} & \scriptsize{Experimentelle} & \scriptsize{Kognitive} & \scriptsize{seminar} &  \scriptsize{Kolloquium} \\
 \scriptsize{Neuro. Grundlagen der KogWis} & \scriptsize{\textcolor{gray}{3 LP}} & \scriptsize{Kognitionswiss.} &\scriptsize{Architekturen}   & \scriptsize{\textcolor{gray}{3 LP}} & \scriptsize{\textcolor{gray}{3 LP}} \\ 
\cline{2-2} \cline{5-6}
 \scriptsize{\textcolor{gray}{6 LP}} & \cellcolor{lightgray} &   &  &  & \cellcolor{lightgray} \\
 & \cellcolor{lightgray} & \scriptsize{\textcolor{gray}{6 LP}}& \scriptsize{\textcolor{gray}{6 LP}} & \scriptsize{Psychophysics \&} & \cellcolor{lightgray} \\
 & \cellcolor{lightgray} & & & \scriptsize{Modelling}& \cellcolor{lightgray} \\
\cline{1-4} \cline{6-6}
 \cellcolor{lightgray} & \cellcolor{lightgray} & \scriptsize{Allgemeine} & \scriptsize{Forschungs-} & & \cellcolor{lightgray} \\
 \cellcolor{lightgray} & \cellcolor{lightgray} & \scriptsize{Psychologie B} & \scriptsize{seminar} & \scriptsize{\textcolor{gray}{6 LP}} & \cellcolor{lightgray} \\
 \cellcolor{lightgray} & \cellcolor{lightgray} & \scriptsize{\textcolor{gray}{3 LP}} & \scriptsize{\textcolor{gray}{3 LP}} &  & \cellcolor{lightgray} \\ 
 \hline \hline
 30 LP & 27 LP & 33 LP & 30 LP & 33 LP & 27 LP \\
 \hline
\end{tabular} 

