\subsection*{KogWiss-FAQ}
Die meisten Fragen rund ums Studium lassen sich durch das KogWiss-FAQ beantworten. Dieses findet ihr unter \url{https://uni-tuebingen.de/de/79193}.	%TODO insert \link{}{}?

\subsection*{Studentische Beratung}
Solltet ihr Fragen haben, welche sich nicht über das KogWiss-FAQ haben klären lassen oder ihr Anliegen habt, die ihr persönlich besprechen wollt, dann könnt ihr euch gerne und jederzeit an die studentische Studienberatung wenden. Hier werdet ihr von kompetenten Kogni-Studierenden zu allgemeinen Fragen des Kogni-Studiums beraten. Diese haben einen direkten Draht zur Fachschaft, Frau Seibold und Frau Hein.  Momentan sind die Ansprechpartner \studBeratung. Ihr könnt sie über die E-Mail Adresse \email{studienberatung@kogwis.uni-tuebingen.de } erreichen. Scheut euch nicht, eine Mail zu schreiben!

\subsection*{Fachschaft Kognitionswissenschaft}
Wir kennen vielleicht nicht auf jede Frage die richtige Antwort, aber meistens kennt jemand jemanden, der zumindest weiß, an wen man sich wenden muss. Also nicht verzagen, Fachschaft fragen ;)

\subsection*{Koordination und Lehre}
Frau Seibold ist für die Koordination und Lehre in Kognitionswissenschaft verantwortlich und ist damit unsere Ansprechpartnerinnen, wenn es um organisatorische Fragen zum Kogni-Studium geht. Ihre Sprech- und Telefonzeiten könnt ihr der Website des Studienbüros entnehmen\footnote{\url{https://uni-tuebingen.de/de/135985}}.

\subsection*{Studienfachberatung}
Für sonstige Fragen zum Studium, könnt ihr euch an Frau Hein wenden. Anfragen werden über die E-Mail \email{studienberatung@kogwis.uni-tuebingen.de} gestellt. Neben der Beschreibung eures Anliegens solltet ihr euren Studiengang, Semester und Matrikelnummer angeben. Termine können danach per E-Mail vereinbart werden.

\subsection*{Prüfungsausschuss} 
Der Prüfungsausschuss ist für die Anrechnung von Prüfungsleistungen zuständig. Macht ihr beispielsweise ein Auslandssemester und wollt eine Veranstaltung, die ihr dort gehört habt, als eine Pflichtvorlesung hier anrechnen, so müsst ihr dies beim Prüfungsausschuss beantragen. Habt ihr bereits einmal studiert und möchtet dort gehörte Vorlesungen anrechnen lassen, geht der Antrag ebenfalls an den Prüfungsausschuss. Den Prüfungsausschussvorsitzenden solltet ihr nur kontaktieren, wenn ihr die anderen Möglichkeiten schon ausgeschöpft habt (zum Beispiel die studentischen Vertreter der Fachschaft gefragt) oder diese euch empfohlen haben, ihn zu kontaktieren. Fragen zur Anrechnung von Veranstaltungen können an \email{anerkennung@kogwis.uni-tuebingen.de} geschickt werden.

\subsection*{Prüfungssekretariat}
Das Prüfungssekretariat ist ausschließlich für die Verbuchung eurer Noten und das Ausstellen von Studienverkaufsbescheinigungen ("Transcript of Record") zuständig. Aktuell ist Frau Gold eure Ansprechpartnerin. Viele Fragen zu Prüfungsanmeldungen und -abmeldungen tauchen immer wieder auf und stehen daher in den Kogni-FAQ. Bevor ihr das Prüfungssekretariat fragt, prüft am besten, ob die Frage bereits irgendwo beantwortet ist. Die Kommunikation mit dem Prüfungssekretariat erfolgt am besten per Mail. Wenn man doch einen Termin ausmachen muss, hängen ab ca. eine Woche vorher Anmeldelisten an der Tür des Servicebüros (Keplerstr. 2, Raum 207). Per E-Mail erreicht man das Prüfungsamt unter \email{pruefungsamt.psychologie@uni-tuebingen.de} \\
Für Formulare, die das Prüfungssekretariat von euch braucht, sei euch der Briefkasten vor dem Büro nahegelegt. Im Normalfall hat man aber mit dem Prüfungssekretariat nicht viel zu tun, da die Noten meistens direkt von den Dozenten an das Sekretariat weitergeleitet werden. 

\subsection*{Psychosoziale Beratung}
Vom Studierendenwerk wird eine psychotherapeutische Beratung\footnote{\url{http://www.my-stuwe.de/beratung-soziales/psychotherapeutische-beratung/}} angeboten. Themen einer solchen Beratung können ganz verschieden sein. Das Angebot ist kostenlos, bzw. wird über den Verwaltungsbeitrag miterhoben, und steht allen Studierenden offen.	%TODO insert \link{}{}?

