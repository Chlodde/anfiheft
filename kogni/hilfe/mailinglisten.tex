Wir betreiben für euch einige Mailinglisten, die euch die Möglichkeit bieten sollen, Fragen an andere Studierende der Kognitionswissenschaft zu richten, deren Fragen zu beantworten und sonstige Informationen weiterzuleiten.  Jeder ist dazu eingeladen die Mails, welche an diese Listen gehen zu lesen und selbst an eine Liste zu schreiben. Wir empfehlen euch grade für die erste Zeit eine Anmeldung in den folgenden Listen:\\

%%\newpage

\newcommand{\mlinfoadressen}[1]{
    Anmelden: Leere Mail an {\footnotesize \email{#1-subscribe@fsi.uni-tuebingen.de}} \\
    Abmelden: Leere Mail an {\footnotesize \email{#1-unsubscribe@fsi.uni-tuebingen.de}} \\
    Hilfetext: Mail mit Betreff \emph{help} an {\footnotesize \email{#1-request@fsi.uni-tuebingen.de}}}

\newcommand{\mlpsychoadressen}[1]{
	Anmelden: Leere Mail an {\footnotesize \email{#1-subscribe@fs-psycho.uni-tuebingen.de}} \\
	Abmelden: Leere Mail an {\footnotesize \email{#1-unsubscribe@fs-psycho.uni-tuebingen.de}} \\
	Hilfetext: Mail mit Betreff \emph{help} an {\footnotesize \email{#1-request@fs-psycho.uni-tuebingen.de}}}

\begin{description}

  \item[kogwiss\At fsi.uni-tuebingen.de] ~\\
    Kogwiss ist die Liste für alle Studierende der Kognitionswissenschaft. Hierüber kommen Mails zu interessanten Vorträgen, Einladungen zum Kogni-Stammtisch, Informationen zum Studium und vieles mehr. \textbf{Auf dieser Liste solltet ihr euch unbedingt anmelden!}

    \mlinfoadressen{kogwiss}
    
\item[info-studium\At fsi.uni-tuebingen.de (Informatik)] ~\\
    Info-studium ist die Liste der Informatiker. Hier gehen vor allem Infos drüber, welche den Informatik-Teil eures Studiums betreffen. \textbf{Die Anmeldung auf dieser Liste ist empfehlenswert. Nur hierüber erhaltet ihr Informationen zu Informatikveranstaltungen!}
	  
	  \mlinfoadressen{info-studium}
    
\item[psycho-news\At fs-psycho.uni-tuebingen.de] ~\\
    Hier postet die Fachschaft Psychologie regelmäßig alle Neuigkeiten aus dem Psychologiebereich.  

    \mlpsychoadressen{psycho-news}    
    
    
\item[student\At fs-psycho.uni-tuebingen.de] ~\\
    Diese Liste ist primär für Psychologie-Studierende, es kommen aber gelegentlich interessante Jobangebote und Informationen für Kognis. 

    \mlpsychoadressen{student}

    
  \item[versuche\At fsi.uni-tuebingen.de (Teilnahme an Versuchen)] ~\\
	 Hier bieten Versuchsleiter immer wieder Versuche an. Wir empfehlen euch, diese Liste in den ersten drei Semestern zu abonnieren, bis die Versuchspersonenstunden abgelegt sind.
  
  \mlinfoadressen{versuche}
  
    \item[sport\At fsi.uni-tuebingen.de] ~\\
Auf dem Sand wird zum Beispiel gerne mal Volleyball oder Fußball gespielt. Verabredungen dazu laufen über diese Liste.
  
  \mlinfoadressen{sport}
  
  \item[info-jobs\At fsi.uni-tuebingen.de (Teilnahme an Versuchen)] ~\\
Wer auf der Suche nach einem (Neben-)Job im Informatikbereich ist, sollte sich auf dieser Verteilerliste für Stellenangebote anmelden.
	\mlinfoadressen{info-jobs}
  
  \item[coding\At fsi.uni-tuebingen.de (Hackathons uä.)] ~\\
  Lust mal aus dem Keller raus zu kommen und mit anderen Leuten in einem stickigen Raum sitzen, Pizza essen und ein Wochenende damit verbringen mit Semikolons um sich zu werfen? Dann ist diese Liste für dich interessant.
  
  \mlinfoadressen{coding}
  

\end{description}

Weitere Informationen zu den Mailinglisten findet ihr auf der Seite der Fachschaft Informatik\footnote{\url{https://www.fsi.uni-tuebingen.de/studium/mailinglisten}}.	%TODO insert \link{}{}?

Für die Experten: Um Mails von diesen Listen zu filtern nutzt man am besten den 'List-Id:'-Header.

\vfill
