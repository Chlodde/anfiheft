\textbf{Morgenstelle}\\
Auf der Morgenstelle befindet sich das Hörsaalzentrum, das Gebäude der Universität, welches sowohl die meisten als auch die größten Hörsäle beherbergt. Hier werdet ihr unter anderem auch die meisten eurer Grundvorlesungen haben.\\
Die Morgenstelle erreicht man am bequemsten mit dem Bus. In 98\% der Fälle solltet ihr an der Haltestelle \emph{BG Unfallklinik} aussteigen. Die anderen Haltestellen \emph{Auf der Morgenstelle} und \emph{Botanischer Garten} eignen sich nur, wenn ihr z.B. in N10 oder N11 müsst.\\ Bei den Hochhäusern auf der Morgenstelle gilt: Jedes Gebäude besitzt einen Buchstaben, der in mehr oder weniger gut sichtbarem Orange außen am Gebäude angebracht ist.  Die Raumnummern orientieren sich an den Gebäudebuchstaben und Stockwerken, z.B. befindet sich der Raum C9A03 im C-Bau im 9. Stockwerk.\footnote{Im Gegensatz dazu ist der Raum 7E02 NICHT im E-Bau im 7. Stockwerk. Achtet auf die Reihenfolge!} Wenn ihr Gebäude C und D betretet, befindet ihr euch bereits im dritten Stockwerk. Warum man sich das so ausgedacht hat, wissen wir auch nicht. Wichtig sind im C-Bau vor allem die Poolräume, welche sich im zweiten Stockwerk links vom Aufzug befinden.

\textbf{Sand}\\
Das Wilhelm-Schickard-Institut für Informatik oder auch einfach nur Sand genannt ist das Institut der Informatik in Tübingen. Auch wenn das Institut etwas außerhalb der restlichen Universität liegt, besticht es durch sein großzügiges Platzangebot und zahlreiche Möglichkeiten zur Freizeitgestaltung in Pausen. Unter anderem befinden sich auf dem Gelände ein Volleyballfeld, ein Fußballfeld und eine Tischtennisplatte. Bälle und Tischtennisschläger liegen im Fachschaftsraum der Fachschaft Informatik und dürfen gerne ausgeliehen werden. Es gibt keine Mensa, jedoch fährt ein bis zwei mal in der Woche ein Foodtruck zum Sand.

\textbf{Psychologisches Institut}\\
Das Psychologische Institut (PI) ist wie der Name schon sagt das Institut der Psychologen. In ihm befindet sich auch der Hörsaal in dem ihr die Vorlesung ''Einführung in die Kognitionswissenschaft'' haben werdet. Außerdem finden hier viele Versuche statt. Manchmal wird das Gebäude auch auf Grund seiner früheren Verwendung \textit{Alte Frauenklinik} genannt.

\textbf{Kupferbau}\\
Der Kupferbau ist das Hörsaalzentrum im Tal. In ihm befinden sich mehrere größere Hörsäle. Hier finden zum Teil Psychologie Veranstaltungen statt, die zu groß für den Hörsaal im Psychologischen Institut sind, oder auch das Unikino.

\textbf{Neue Aula}\\
Die neue Aula im Tal ist mit den beiden Brunnen das Vorzeigegebäude der Uni. Hier finden vor allem Jura-Vorlesungen statt, jedoch wird der Hörsaal Audimax sowie der Festsaal (bei im ersten Stock) gerne für größere Veranstaltungen verwendet.

\vfill
