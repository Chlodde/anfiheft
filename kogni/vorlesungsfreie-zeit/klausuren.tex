% \coronabox{Aufgrund der Corona Pandemie und den einhergehenden Hygiene-Vorschriften, kann über das Format der Klausuren noch nichts gesagt werden, da sich die Regelungen oft ändern. Dies wird dann in den einzelnen Veranstaltungen besprochen. Es ist jedoch wahrscheinlich, dass einige Prüfungen nicht wie sonst in präsenz, sondern online stattfinden.} \\
Die größte Freude der vorlesungsfreien Zeit stellen die Klausuren dar. In ihnen müsst ihr zeigen, was ihr das Semester über gelernt habt. Die Klausuren in Tübingen finden zumeist in der letzten Woche der Vorlesungszeit und der ersten Woche der vorlesungsfreien Zeit statt. Solltet ihr durch eine der Klausuren fallen oder euch entscheiden, dass ihr für eine Klausur noch nicht genug gelernt habt, finden Nachholtermine zu den Klausuren in den letzten beiden Wochen der vorlesungsfreien Zeit statt. Ihr könnt euch also bei den meisten Klausuren entscheiden, ob ihr sie zu Beginn oder zum Ende der Ferien schreibt. Aber vorsicht: besteht ihr die Klausur am späteren Termin nicht, könnt ihr sie je nach Vorlesung erst ein Jahr später erneut schreiben.\\
\\
Wichtig bei den Klausuren ist, dass ihr euch rechtzeitig dafür anmeldet. Die genauen Anmeldemodalitäten werden euch normalerweise in den ersten Vorlesungen erläutert. Im Zweifel ist es ratsam, seinen Dozierenden rechtzeitig nochmal zu fragen. Bei den Informatik-Veranstaltungen kann man sich oft in ALMA unter 'Prüfungsverwaltung' oder in der Verwendeten Lehrplattform anmelden. Zu den Psychologie-Klausuren gehen meistens Listen rum und es liegen zusätzlich Anmeldelisten im psychologischen Institut vor dem Raum auf 4.432 auf Ebene 4.
