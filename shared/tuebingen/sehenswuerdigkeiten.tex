\begin{description}
\item[Tübinger Schloss] Eines der Wahrzeichen Tübingens ist sein Schloss. Hat man es erst mal geschafft, den Schlossberg hochzulaufen, bietet sich einem ein grandioser Blick über Tübingen. Im Schloss ist außerdem das Museum der Universität Tübingen (MUT) untergebracht, welches immer wieder Interessante Ausstellungen aus den Schätzen der Universität zeigt. Für Tübinger Studierende ist der Eintritt kostenlos.

\item[Schönbuch] Der Schönbuch ist ein großes Waldgebiet nördlich von Tübingen. Es eignet sich sowohl für kleine Spaziergänge als auch ausgedehntere Wanderungen.

\item[Stocherkähne]
Gerade im Sommer sieht man immer wieder Stocherkähne auf dem Neckar. Bei den Stocherkähnen handelt es sich um lange Boote, welche mittels eines langen Stocks durch das Wasser geschoben werden. Bei der Touristeninformation kann man Fahrten buchen, bei der man gleich noch zahlreiche Informationen über Tübingen erzählt bekommt. Wer nicht so viel Geld hat und es sich zutraut, kann auch selbst einen Stocherkahn mieten und fahren. Die Studierendenwohnheime WHO und Hindenburgkaserne haben beispielsweise einen Kahn, welchen man als Studierender günstig mieten kann.

%\item[Stiftskirche]
%Die Stiftskirche liegt zentral in der Tübinger Innenstadt. Neben ihrem imposanten Anblick bietet ihr Turm auch einen großartigen 360-Grad-Ausblick über Tübingen. 

\item[Bebenhausen]
Ein Stück von Tübingen entfernt liegt der kleine Ort Bebenhausen mit seinem mittelalterlichen Kloster. Von Tübingen kann man gut nach Bebenhausen laufen. Sollte einem nach der Klosterbesichtigung der Weg zurück zu weit sein, kann man auch den Bus zurück nach Tübingen nehmen.

\item[Wurmlinger Kapelle]
Richtung Rottenburg erhebt sich auf einem aus allen Richtungen gut sichtbaren Hügel die Wurmlinger Kapelle. Auch diese kann von Tübingen über den Rücken des Schlossbergs gut erwandert werden. Auch hier bietet sich einem ein grandioser Blick über das Ländle.

%\item[Neckar(-insel)]
%Mitten durch Tübingen fließt der Neckar. In und um ihn trifft man sich gerne zum Chillen oder Grillen. Auf der Neckarinsel gibt es ausreichend Rasenfläche, um seine Picknickdecke auszubreiten. Wer keine dabei %hat, kann auch einfach mit einem Bier auf der Neckarmauer Platz nehmen. Diese erstreckt sich, 4m über den Neckar erhebend, an der Innenstadtseite des Neckars entlang.

\item[Burg Hohenzollern]
Die Burg Hohenzollern, auch Schloss Neuschwabenstein genannt\footnote{Sagt eigentlich niemand, fanden wir aber witzig.}, ist ein neugotisches Schloss auf der Schwäbischen Alb.

\item[Stadtmuseum]
Im Stadtmuseum finden immer wieder Ausstellungen zur Geschichte Tübingens statt. Mit dem Gutschein aus eurem Begrüßungsheft der Stadt kommt ihr sogar einmal kostenlos ins Museum, der Eintritt ist aber auch ansonsten nicht teuer.

\item[Die schwäbische Alb]
Südlich von Tübingen beginnt die schwäbische Alb. Hier kann wunderbar gewandert werden, auf der schwäbischen Alb kann man auch immer wieder unberührte Flecken Natur finden.

\end{description}
