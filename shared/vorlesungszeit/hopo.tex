An der Uni mag es zwar manchmal so wirken, dass die Studierenden der unwichstigste
Teil des Apparates sind.
Aber für uns Studierennde ist das eben nicht der Fall.
Damit das auch die anderen Instanzen und Interessensgruppen verstehen, gibt es
etwas das nennt sich Hochschulpolitik und \textit{Verfasste
Studierendenschaft}. Dort kann jeder Studierende sich engagieren oder seine/ihre
Anliegen hintragen.

Was ist denn die Verfasste Studierendenschaft? Du und alle anderen Studierenden
Tübingens. So einfach die kurze Antwort. Die Lange bezieht sich auf einen
Ausschnitt des \textit{Landeshochschulgesetzes}.  Zusammengefasst steht da
sowas wie ,,Studierende an einem Hochschulstandort haben das Recht sich zusammen
zu schließen und als Interessensvertretung in der Hochschule gehör zu finden''.
Das Interessante ist, dass die Verfasste Studierendenschaft auch Gelder bekommt,
beispielsweise aus dem Semesterbeitrag.

In Tübingen kann man mehr als 200 Dinge studieren und etwa ein Drittel der
Einwohner sind Studierende. Es wundert also nicht, dass es hier eine
gut aufgestellte Studierendenvertretung gibt.


\subsection{StuRa, AStA, was das?}

Vielleicht sagt euch der Begriff AStA -- also Allgememeiner
Studierendenausschuss -- etwas. In Tübingen wurde der vor vielen Jahren vom
StuRa -- Kurzform von Studierenden Rat -- entmachtet und seitdem erfüllt dieser
die Rolle der allgemeinen Studierendenvertretung. Während wir als Fachschaft
eine studentische Vertretung im Fach sind, ist der StuRa allgemeingültiger.

Beispielsweise gibt es verschiedenste Arbeitskreise. Von Politische Bildung,
Umwelt, Gleichstellung und Familienfreundliche Hochschule zu Ract!, überregionale
Zusammenarbeit. Jeder Studierende kann sich dort direkt selber einbringen.
Eine Liste findet ihr auf der Seite des StuRa
\footnote{\url{https://stura-tuebingen.da/arbeitskreise}}. Schaut da doch mal
rein. Oder informiert euch auf den Social Media Kanälen 
\faIcon{instagram}\href{https://www.instagram.com/vs\_tuebingen}{\texttt{vs\_tuebingen}}
und \faIcon{mastodon}\href{https://toot.kif.rocks/@vs@tuebingen.network}{\texttt{@vs@tuebingen.network}}.


Außerdem vergibt der StuRa ein Notlagenstipendium an Studierende in Not, fördert
Veranstaltungen für Studierende und hält jährlich eine studentische Vollversammlung
ab, in der alle Studierende das Recht haben Anträge zu stellen und über diese abzustimmen.

Jedes Sommersemester werden die ca. 20 Vertreter für das nächste Jahr gewählt.
Wählen und sich aufstellen dürfen alle immatrikulierten Studierende Tübingens,
wobei es eine Listenwahl ist, das heißt verschiedene
Hochschulgruppen melden eine Liste zur Wahl an. Solche Hochschulgruppen
sind beispielsweise die Grüne Hochschulgruppe (GHG), die Juso Hochschulgruppe
oder die FSVV. An den Namen sieht man bereits, dass das mehr als nur eine
SMV\footnote{Schüler Mit Verantwortung} an der Uni ist und es doch schon
allgemeinpolitischer ist.

\link{https://stura-tuebingen.de/}{Studierendenrat Tübingen}


\subsection{FSVV}
Was ist denn das jetzt wieder für eine Abkürzung? FSVV steht
für \textbf{F}ach\textbf{S}chaften\textbf{V}oll\textbf{V}ersammlung und ist
quasi die Partei aller Fachschaften. Was heißt, dass wir als Fachschaft in den
wöchentlichen Versammlungen dieses Bündnisses Stimmrecht besitzen.

