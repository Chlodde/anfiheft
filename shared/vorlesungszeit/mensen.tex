%\coronabox{Momentan\footnote{Stand: 30.03.2022} sind alle Mensen geöffnet. Dies
%kann sich aber je nach Pandemielage wieder ändern. Ähnliches gilt bei den
%Cafeten.}

In Tübingen gibt es zwei große und eine kleinere Mensa, welche vom
Studierendenwerk betrieben werden. Hier zahlt ihr normalerweise bargeldlos mit
eurem Studierendenausweis.\\ Die Öffnungszeiten der Mensen sind:
\begin{center}

Mensa Morgenstelle:\\
Mo - Fr~11$^{\underline{30}}$~--~14$^{\underline{00}}$

\bigskip

Mensa Shedhalle:\\
Mo - Do~11$^{\underline{15}}$~--~14$^{\underline{00}}$

\bigskip

\nopagebreak
Prinz Karl:\\
Mo - Fr~11$^{\underline{00}}$~--~15$^{\underline{00}}$\\
Essensausgabe: ~11$^{\underline{15}}$~--~15$^{\underline{00}}$

\end{center}

Die Speisepläne findet ihr auch auf der
Internetseite\footnote{\url{https://www.my-stuwe.de/mensa/}} des
Studierendenwerks oder in der \emph{my-stuwe}-App.\\

Neben den drei Mensen gibt es noch weitere Cafeterien mit unterschiedlich
großem Angebot. Auf der Morgenstelle gibt es beispielsweise in der Cafeteria
unter der Mensa neben belegten Brötchen sowie Kaffee und Kuchen auch Burger,
Pommes oder Currywurst. Für den schnellen Kaffee in der Vorlesungspause eignet
sich vor allem die Cafeteria im Hörsaalzentrum.\\ Neben dem kulinarischen
Aspekt bieten die Cafeterien häufig auch Arbeitsplätze mit Steckdosen und
WLAN.\\
%\vfill \pagebreak
In Tübingen gibt es folgende Cafeterien:
\begin{itemize}
	\item Cafeteria Unibibliothek
%	\item Cafeteria Wilhelmstraße
	\item Cafeteria Morgenstelle (im Mensagebäude)
	\item \textcolor{gray}{Cafeteria Hörsaalzentrum} (Die Flachpfeifen haben die einfach geschlossen!)
	\item Cafeteria Prinz Karl
	\item Cafeteria Clubhaus
	\item Cafeteria Theologicum
	\item Cafeteria Neuphilologicum (Brechtbau)
\end{itemize} Die Öffnungszeiten der Cafeterien findet ihr ebenfalls auf der
Seite des Studierendenwerks.
