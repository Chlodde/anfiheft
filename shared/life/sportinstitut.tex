
Das Sportinstitut erstellt für jedes Semester und für jede
  vorlesungsfreie Zeit ein ausführliches Sportprogramm, welches sowohl in
  den Hallen (Wilhelmstraße, Alberstraße) ausliegt als auch auf der
  Seite des Sportinstituts\footnote{
  \url{https://uni-tuebingen.de/de/174006}}
  zu finden ist. Das breit gefächerte Angebot enthält neben den "`normalen"', regelmäßig stattfindenden Individuell- und Teamsportarten auch einige eher spezielle Sportarten (z.B. Aqua-Jogging oder Ultimate Frisbee) sowie Exkursionen (z.B. Kajak- oder Skitouren). Ein besonderer Schwerpunkt scheint auf einem Sortiment teils ein wenig esoterischer Kampfsportarten zu liegen. Für einige wenige Sportarten wie z.B. Segeln oder Paragliden gibt es zudem Kooperationen mit professionellen Anbietern. Das Sportinstitut betreibt auch einen Kraftraum mit Kletterturm sowie eine Sauna. Für Studierende ist
  die Teilnahme an Kursen normalerweise sehr günstig. 
  Desweiteren ist beispielsweise die Nutzung der Sportanlagen kostenlos, wer also gerne mal beispielsweise Lauftraining absolvieren möchte, kann das beim Sportinstitut auf der Laufbahn gerne tun, nachdem er eine Track-Karte im Hochschulsport-Sekretariat (Passbild mitbringen!) abgeholt hat.
  Die Kurse werden von qualifizierten Sportdozenten durch\-ge\-führt.\medskip

Vorsicht: Im Umkleidebereich wird "`traditionell"' extrem viel
  geklaut, also Uhr, Portemonnaie und am besten auch das Smartphone gleich
  zu Hause lassen und die Sporttasche wäh\-rend des Trainings in
  die Halle mitnehmen. Den Studierendenausweis solltet ihr jedoch immer dabei haben, denn der wird
  ab und zu am Eingang kontrolliert.
