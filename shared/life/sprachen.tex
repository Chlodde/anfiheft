Und wenn wir schon gerade beim Ausland sind: Man kann in Tübingen
  natürlich auch als Informatiker Fremdsprachen lernen.  Die besten
  Möglichkeiten hierzu sind das Fachsprachenzentrum der
  Universität (FSZ) und das Deutsch-Amerikanische Institut.


Am Fachsprachenzentrum\footnote{\url{https://www.uni-tuebingen.de/fsz/}} finden
  semesterbegleitende Kurse und während der Semesterferien zwei- bis
  vierwöchige Intensivkurse in den wichtigsten Sprachen statt:
  Englisch, Französisch, Italienisch, Russisch, Spanisch, Japanisch und Schwedisch (vermutlich auf Grund unserer Partneruni in Uppsala). Neben den
  Einsteiger- und Anfängerkursen (UNIcert I/II mit je drei
  Niveaustufen) kann man auch fachspezifische Kurse belegen
  z. B. Beruf und Studium, Naturwissenschaften,
  Wirtschaftswissenschaften, etc.  Ob man direkt mit einer höheren
  Niveaustufe beginnen darf (oder im Grundkurs beginnen muss)
  entscheidet ein Einstufungstest vor Ort.  Gegeben werden die Kurse
  meist von Muttersprachlern und haben mittleres bis hohes Niveau.
  Die Einschreibung zu den semesterbegleitenden Kursen ist dieses Semester normalerweise am 7.10. um 9 Uhr, die Zeiten werden im
  Internet (s. o.) bekannt gegeben.  Gerade bei begehrten Kursen
  (d. h. fast alle Einsteigerkurse wie z. B. Spanisch I oder
  Italienisch I) sollte man aber bei der  Einschreibung schnell sein.
  Meist sind alle Plätze schon am ersten Einschreibetag vergeben. Seitdem die Anmeldung online stattfindet sind die langen Schlangen vor dem FSZ jedoch Vergangenheit. Dafür kann der Lieblingskurs bereits um 9:20 vollständig ausgebucht sein.\\
  Die Kurse sind für Studenten kostenlos, sie werden über Studiengebühren finanziert. Wer sich angemeldet hat muss jedoch auch erscheinen, da Anwesenheitspflicht besteht. Ansonsten droht eine Sperre für weitere Kurse am FSZ für das kommende Semester in der jeweiligen Sprache. Sollte man nicht in den Kurs hineingekommen sein, sollen es manche schon mit persönlicher Anwesenheit zur ersten Stunde geschafft haben, den Dozenten zur Aufnahme von noch einem Studenten zu überreden oder diesen Studenten der Warteliste vorzuziehen. Allerdings muss man beachten, dass im ersten Semester grundsätzlich sehr viel auf euch zu kommt, überschätzt euch nicht.\\

Früher konnte man über die Uni auch relativ leicht an den Kursen der entsprechenden Studiengänge wie Italienisch, Spanisch etc. teilnehmen. Viele von diesen sind inzwischen jedoch wegen IVWL etc so weit von eigenen Studenten überfüllt, dass diese Möglichkeit rar geworden ist. Französisch und Spanisch sind für externe Studenten komplett geschlossen. Einige Fachbereiche wie z.B. die Sinologie bieten jedoch auch Kurse für interessierte Studenten anderer Fachrichtungen an.  Generell sind die Kurse der "`richtigen"' Fachbereiche für ein hohes Niveau bekannt.
  
Eine Alternative zu alle dem ist natürlich die klassische Volkshochschule.  Ihr Angebot kann
  im Internet\footnote{\url{https://www.vhs-tuebingen.de/}} eingesehen
  werden.  Im Angebot der vhs finden sich Kurse vieler Sprachen, von Englisch bis Hindi oder Suaheli in jeweils mehreren Niveaustufen und mit vielen
  Spezialkursen wie Konversationskurse, Grammatikkurse, Reisekurse,
  Kinderkurse, etc.  Sie kosten zwischen 90 und 150 EUR (je nach
  Sprache, Kursart, Dauer, ...) - zumeist sind es jedoch etwa 90 EUR.  Das Niveau dieser Kurse ist
  im Allgemeinen relativ niedrig, allerdings kann man das
  auch vorher telefonisch bei der vhs erfragen und sich dort beraten
  lassen.
