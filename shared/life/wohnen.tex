% life/wohnen.tex
Wohnen in Tübingen ist ein großes Thema.  Wohnheime werden von
  verschiedenen Institu\-tionen betrieben: Studierendenwerk AdöR,
  Tübinger Studentenwerk e.V., einigen christlichen Vereinigungen und
  sonstigen Vereinen.  Angeblich hat Tübingen sogar die höchste
  Versorgungsquote mit Wohnheimplätzen in ganz Baden-Württemberg.
 Tendentiell ist es keine allzu schlechte Idee sich für die ersten Semester einen Wohnheimsplatz
 zu suchen und erst danach eine WG oder  "Ahnliches zu gr"unden. Das funktioniert
 in der Regel besser als gleich im 1. Semester WGs zu gr"unden, bei denen dann die H"alfte der Mitglieder im
 2. Semester nicht mehr da sind.

Bei der privaten Wohnungssuche läuft sehr viel über das Internet. Hier haben sich \url{https://www.wg-gesucht.de} und \url{https://www.zwischenmiete.de} als sehr nützlich erwiesen.

Die Wohnheimplätze sind im Vergleich zu vielen privaten
  "`Wohn-Angeboten"' etwas günstiger.  Euer ganzes Studi-Leben könnt
  ihr jedoch nicht im selben Wohnheim verbringen, denn üblicherweise
  setzen die Träger Fristen von 4 bis 6 Semestern.

Bei der Suche nach einem Zimmer oder einer Wohnung bei privaten
  Vermietern kann die Zimmervermittlung des Studierendenwerks helfen.
  Sprechstunden sind Mo.--Fr.,~10:00--16:30~Uhr in der Mensa Wilhelmstraße.
  Die Vermittlung der Zimmer ist kostenlos; allerdings müsst ihr persönlich vorbeikommen.
  Die Neuzugänge  eines Vor- bzw. Nachmittags werden gesammelt, und jeweils um
  11:30 bzw. 14:00~Uhr werden dann die
  Adressen gegen ein Pfand von 10,00 EUR an die Zimmersuchenden
  ausgehändigt.
  
Gerade zu Semesterbeginn ist die Zimmervermittlung oft überlaufen.
  Außerdem sind richtig gute Angebote äußerst rar, denn diese
  gehen, längst bevor sie den Weg zur Zimmervermittlung gefunden
  haben, unter der Hand weg.  Wenn ihr also niemanden kennt, dessen
  Bruder eine Freundin hat, deren Studienkollegin mit einem Typ
  zusammenwohnt, der ..., dann bleibt euch eigentlich nur die
  Möglichkeit, Mittwochs und Samstags früh aufzustehen und euch
  als erster in ganz Tübingen den Wohnungsmarkt im Schwäbischen
  Tagblatt reinzuziehen.  Oder ihr investiert gleich in eine eigene
  Anzeige;  am besten in der Samstagsausgabe.

In den Mensa Wilhelmstraße und der Mensa
  Morgenstelle liegt (kostenlos!) das \emph{Dschungelbuch} aus, das
  u.a. Informationen zu den Wohnheimen enthält. Sehr empfehlenswert!
\enlargethispage{1ex}

Kurzfristig eignet sich auch die Jugendherberge in der Gartenstraße 22, Tel: 07071/23002
