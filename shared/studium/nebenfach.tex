
Wenn ihr einen Bachelor of Arts anstrebt und Informatik als Nebenfach gewählt habt, seid ihr hier richtig. Ihr müsst vier Module abdecken, dabei gelten folgende Regeln:

\begin{itemize}

	\item Im \textbf{Pflichtbereich A} müsst ihr Informatik I belegen
	\item Im \textbf{Wahlpflichtbereich B} habt ihr die Wahl zwischen Informatik II und Theoretischer Informatik
	\item 20 LP braucht ihr für den \textbf{Wahlpflichtbereich C}. Hier habt ihr Einführung in die technische Informatik, Informatik der Systeme, das Basispraktikum, Algorithmen und Mathematik I zur Auswahl, außerdem die Vorlesung aus dem Wahlpflichtbereich B, die ihr noch nicht gehört habt.
	\item Als letztes gibt es hier den \textbf{Wahlpflichtbereich D}. Hier müsst ihr 24 LP holen und
	habt dafür alle noch nicht belegten Vorlesungen aus der Informatik zur Auswahl.
\end{itemize}
