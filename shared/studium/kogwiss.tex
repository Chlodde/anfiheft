
Der Studiengang Kognitionswissenschaft hat seinen Ursprung in der ehemaligen
Fakultät für Informations- und Kognitionswissenschaften und ist als
Gemeinschaftsprojekt der Informatik und Psychologie und weiteren Fächern
entstanden.

Ziel ist es den Fluss und die Verarbeitung der durch die Sinnesorgane gewonnenen
Informationen hin zu einer Aktion zu untersuchen. Auch Formen der Kommunikation,
insbesondere die Sprache, können eine solche Aktion sein. Da sich diese
Disziplin aus der Informatik entwickelt hat ("`wir bilden einen Menschen 
nach"'), werden nach wie vor insbesondere Methoden der Informatik benötigt - 
nicht zuletzt auch um Versuchsanordnungen zu programmieren und entsprechende 
Simulationen durchzuführen. 

Um alle Aspekte der Kognitionswissenschaft erfassen zu können, erwartet euch ein
vielfältiges und auch ein wenig stressiges Programm der Fächer Informatik,
Mathematik, Psychologie, Neurobiologie und auch der Linguistik und Philosophie: 

\begin{itemize}

\item Wie in allen anderen Studiengängen auch müsst ihr \textbf{Informatik I}
    und \textbf{Informatik II} aus dem Bereich der praktischen Informatik
    besuchen. Auch die Vorlesungen \textbf{Mathematik I bis III} werdet ihr
    h\"oren; au\ss{}erdem ein wenig Theorie in der \textbf{Informatik III}
    und im vierten Semester in \textbf{Algorithmen}. Einige zeitaufwändige
    Veranstaltungen der Informatik werden jedoch auch an euch vorüber gehen, wie
    zum Beispiel die technische Informatik und die für den Studiengang eher
    unwichtige Stochastik. Statt dessen werdet ihr aber \textbf{Mathematische 
    Statistik I und II} besuchen, welches aber von der Psychologie ausgerichtet 
    wird und daher sehr auf die Anwendung in der Psychologie ausgerichtet 
    ist. Wenn gewünscht dürft ihr im Zuge des Projektpraktikums auch an 
    ausgewählten Projekten des Programmierpraktikum teilnehmen. Alternativ dazu 
    stehen aber auch noch Projekte aus der Psychologie und Neurobiologie zur 
    Auswahl.

\item Im Bereich der Neurobiologie werdet ihr wie auch die Bioinformatiker im 
ersten Semester \textbf{Neurobiologie und Sinnesphysiologie} (bei denen heißt 
es jedoch "`Tierpysiologie"' - ist aber inhaltlich gleich) besuchen. Damit es 
für euch keine Überschneidungen gibt und auf Grund der steigenden Anzahl von 
Studenten, haben sich die Biologen dazu bereit erklärt diese Vorlesung doppelt 
anzubieten - einmal nur für euch! Zur Vorlesung gehört auch ein gleichnamiges 
Seminar in dem ihr in kleineren Gruppen Vorträge zu verschieden 
neurobiologischen Themen vorbereiten und vortragen werdet.

\item Die Psychologie wird euch fast jedes Semester begleiten. In 
verschiedensten Vorlesungen werdet ihr mit der Denkweise der Psychologen 
vertraut gemacht und erhaltet Einblicke in die verschiedenen Teilbereiche der 
Psychologie. Dabei h\"ort ihr in den ersten vier Semestern zunächst einmal 
die \textbf{Allgemeine Psychologie A bis D} (und damit eine Einf\"uhrung in die 
Themen "`Wahrnehmung"', "`Lernen, Emotion \& Motivation"', "`Aufmerksamkeit \& 
Denken"' und "`Gedächtnis \& Sprache"') sowie \textbf{Forschungsmethoden der 
Psychologie}. Au\ss{}erdem besucht ihr im zweiten Semester die Vorlesung 
\textbf{Biologische Psychologie II}.

\item Mit im Programm ist auch die Vorlesung \textbf{Einführung in die 
Linguistik} im dritten Semester sowie Einf\"uhrungsveranstaltungen in andere 
Bereiche der Lingustik ("`Phonetik/Phonologie"', "`Semantik"', "`Grammar 
Formalisms"', "`Parsing"', "`Statistical Language Processing"' etc.).

\item Ein paar wenige Vorlesungen werden euch auch in den Bereich der 
Philosophie entführen. Dabei dürft ihr euch im dritten und vierten Semester in 
die Vorlesungen der Philosophen setzen. Diese reichen von  \textbf{Einführung 
in die Logik} bis hin zu \textbf{Große Gestalten der griechischen Philosophie}.

\item \textbf{Schlüsselqualifikationen} müssen natürlich wie in jedem 
BSc-Studiengang auch von euch besucht werden.

\end{itemize}

Als Orientierungsprüfung müsst ihr davon "`Informatik I"' und "`Neurobiologie 
und Sinnesphysiologie"' in den ersten beiden Semestern bestehen.

