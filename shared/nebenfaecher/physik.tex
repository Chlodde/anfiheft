Physik ist ein Nebenfach mit mittlerem Aufwand. 
Belegt ihr Physik als euer Schwerpunktmodul habt ihr die Pflichtmodule "`Experimentalphysik I für Naturwissenschaftler und Pharmazeuten"', "`Experimentalphysik II für Naturwissenschaftler und Pharmazeuten"' so wie das "`Physikalische Praktikum für Naturwissenschaftler"' zu belegen. Gegenstand der Experimentalphysik I sind die Grundlagen der Mechanik, Akustik und Wärmelehre. In der Experimentalphysik II geht es dann weiter mit Elektrizitätslehre, Optik und Atomphysik. Im Physikalischen Praktikum bekommt ihr dann die Fähigkeiten vermittelt, Versuche in der Physik durchzuführen und auszuwerten. Der Turnus von Expreimentalphysik I ist jedes Wintersemester, dass von Experimentalphysik II in jedem Sommersemester. Das Physikalische Praktikum gibt es in jedem Semester.
Verantwortlich für das Nebenfach Physik ist Prof. Hauck\footnote{\email{hauck@informatik.uni-tuebingen.de}}.


Die Vorlesung Ex\-pe\-ri\-men\-tal\-phy\-sik wird üblicherweise in zwei
  Schwierig\-keits\-graden angeboten. Einmal in Form einer dreistündigen
  Vorlesung zzgl. einer Stunde Experimente ohne weitere Übungen. Zum
  anderen gibt es eine 6-Stündige Vorlesung zzgl. Übungen, die durch die
  wesentlich höhere Stundenzahl viel mehr ins Detail gehen kann.

Das Praktikum kann viel Spaß machen, wenn ihr ungestresst dran geht
  und beim Experimen\-tieren nicht dauernd auf die Uhr guckt.  Ihr müsst
  als Vorbereitung 2--3 Seiten über die zum entsprechenden Versuch
  im Anleitungsheft aufgeführten Stichworte zu Papier bringen und dann
  Eure Versuche mitprotokollieren.

Man kann das Praktikum entweder über das Semester verteilt in 3 Stunden
  wöchentlich machen oder als Block in den Semesterferien. Für das
  Blockpraktikum sind traditionell mehr Interessenten da, also sei es
  geraten, sich frühzeitig dafür einzuschreiben.
  Wer Physik in der Schule bis zum Schluss als LK hatte, kann auch das Praktikum
  ohne die Vorlesung EP I schon machen, der Stoff hat dort nur unwesentlich
  höheres Niveau.
