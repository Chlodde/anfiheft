% nebenfaecher/vwl.tex
\nebenfachveraltet
\emph{Im Zuge der Umstellung der Wirtschaftswissenschaften auf Bachelor/Master
ist derzeit nicht klar, ob das Nebenfach VWL von (Dipl.) Informatikern belegt werden kann.}\par\bigskip

Wie das Nebenfach BWL wird VWL von der Wirtschaftswissenschaftlichen
Fakultät angeboten. Die allgemeinen Infos, die im vorigen Artikel zum
Nebenfach BWL gegeben wurden, gelten damit auch für die VWLer.

Auf der Homepage der Wirtschaftswissenschaften findet Ihr unter dem Punkt
"`Studium"' ein PDF-Dokument, das alle Informationen zum Nebenfach BWL und
VWL enthält. Die Informationen dort sind ausführlicher und selbstverständlich
verbindlich ;-)

Prinzipiell legt man in VWL die Prüfungen vorlesungsbegleitend ab, d.h.
am Ende jedes Semesters gibt es eine Klausur über die Vorlesung, in der
Ihr grade saßt. Eine Nachklausur zum Ende der Semesterferien wird auch normalerweise
angeboten; ebenso wie eine "`Nachklausur der Nachklausur"' etwa ein halbes Jahr danach. Welche
dieser Klausuren Ihr mitschreibt bleibt Euch überlassen. Wenn man alle vier Klausuren
bestanden hat, hat man das Vordiplom. Anmeldungen für diese Klausuren sind verbindlich; Abmelden ist nicht möglich.
Orientierungsprüfung ist die zweistündige Prüfung in Volkswirtschaftslehre I.
Sie muss bis spätestens nach dem 3. Semester abgelegt sein; es gibt einen Wiederholungsversuch.

Für diese Klausuren müßt Ihr Euch \textbf{rechtzeitig} beim Prüfungsamt
in der Nauklerstraße schriftlich anmelden (Formulare liegen dort aus). Verspätetes
Anmelden ist nicht möglich; beachtet deshalb bitte unbedingt die Termine. Selbige
findet Ihr auf der Homepage der Wirtschafts-Fakultät. Es ist wahrscheinlich, dass
Ihr Euch bereits in den nächsten Tagen oder Wochen angemeldet haben müßt.

Im Grundstudium müßt Ihr in VWL folgende Vorlesungen hören:

\begin{itemize}
\item Einführung in die Volkswirtschaftslehre (VWL I)
\item Mikroökonomik I (VWL II)
\item Makroökonomik I (VWL III)
\item Statistik I
\end{itemize}

Es empfiehlt sich wohl, jedes Semester genau eine dieser Vorlesungen zu besuchen;
beginnen tut man hiermit sinnvollerweise im Wintersemester mit VWL I. Die
Übungen sind nicht verpflichtend, aber durchaus empfehlenswert, und sei es,
um dort die eigenen Mathe-Aufgaben zu lösen.

In VWL gibt es normalerweise zu jeder Vorlesung ein Skriptum, dessen Anschaffung
sich durchaus lohnt. Meistens wird es nur in der ersten Woche ausgegeben (gegen
Bargeld); manchmal hat man Glück und es ist auch im Internet zu finden
 oder im Copyshop zu kaufen.

Der Stundenplan für das erste Semester mit Nebenfach VWL erweitert
sich also um:

\begin{itemize}
\item Einführung in die Volkswirtschaftslehre
bzw.
\item Statistik I
\end{itemize}

(Die Zeiten u. Professoren waren bei Druck leider noch nicht bekannt;
Genaueres hängt aber zu Semesterbeginn am schwarzen Brett in der
Mohlstraße 36 aus bzw. findet sich im Internet \url{http://www.wiwi.uni-tuebingen.de/}.)
