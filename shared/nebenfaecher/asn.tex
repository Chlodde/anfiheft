% nebenfaecher/asn.tex
% kontrolliert: Oktober 2001

\textit{Die Informationen in diesem Artikel zum Nebenfach am Seminar für Sprachwissenschaft sind leider veraltet. Bitte erkundigt euch direkt dort, ob dieses Nebenfach noch so angeboten wird.}

Das Nebenfach Linguistik (auch: Computerlinguistik, Textwissenschaft,
  Sprachwissenschaft) kann auf zwei Arten studiert werden.  Erstens am
  Arbeitsbereich Textwissenschaft innerhalb der Fakultät für
  Informatik (für nähere Informationen Prof. Schweizer oder das
  WWW\footnote{\url{http://www-ct.informatik.uni-tuebingen.de/}}).  Zweitens im
  Rahmen der Fakultät für Neuphilologie.  Nur für diese zweite
  Möglichkeit gelten die folgenden Informationen.\bigskip

Das Nebenfach Linguistik wird von der Neuphilologischen Fakultät
  ausgerichtet, genauer gesagt vom Seminar für Sprachwissenschaft (SfS).
  Adresse: Wilhelmstraße 113 (gegenüber vom Sportinstitut).
Dort wird ein Studiengang "`Allg. Sprachwissenschaft und Nebenfächer"'
  (ASN) angeboten.
Dieser besteht aus Allg. Sprachwissenschaft, Informatik und
  Psychologie.
Computerlinguistik wäre eigentlich ein besserer Name.

Für Informatiker besteht das Nebenfach Linguistik aus dem
  reduzierten ersten Teil des ASN-Studiengangs.
Das heißt für das Grundstudium:
  Proseminar Einführung in die Linguistik, Proseminar Syntax und
  Proseminar Semantik.

Die drei Scheine können problemlos in vier oder weniger Semestern
  gemacht werden.  Es kann auch noch im zweiten oder dritten Semester zu
  Linguistik gewechselt werden und bis zum Vordiplom das Pensum
  geschafft werden.
  Die Note ergibt sich als arithmetisches Mittel aus den 3 Scheinen.
  Fragen beantwortet Fritz Hamm (Do, 16--17 Uhr, Tel. 29-77305).\bigskip

Im Hauptstudium benötigt man lediglich 2 Hauptseminar--Scheine. Für
  Informatiker bieten sich da naturgemäß die beiden Scheine
  Computerlinguistik I und II an, die regelmäßig gelesen werden.  Es sind
  auch andere Themen wie z.B. Phonologie denkbar, das wurde aber bisher von
  niemandem so gemacht.  Bei Interesse sich vorher absichern, ob das nachher
  auch für das Nebenfach--Diplom gilt! Über die beiden erzielten Scheine
  legt man dann eine 30-minütige mündliche Prüfung ab.

Darüber hinaus gibt es noch einen offiziellen Studienberater im SfS.
  Momentan ist das Dale Gerdemann (Tel. 29-74967).
  Zu empfehlen sind auch die Schwarzen Bretter:
  Die umfangreichsten Informationen hängen im SfS selbst, aber das
  Wesentliche (z.B. Vorlesungstermine, Seminare etc.) findet ihr auch im Brecht-Bau (Neuphilologikum).




