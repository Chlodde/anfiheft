% Nebenfaecher/allgemeines.tex

\newcommand{\nebenfachveraltet}[0]{\emph{Achtung: Dieser Abschnitt ist möglicherweise veraltet.}\par\bigskip}

Für alle, an denen es bisher vorbei gelaufen ist: \textbf{Nur Informatik
  studieren geht nicht; jeder Info-Studi braucht einen Schwerpunktbereich} (außerhalb der Prüfungsordnung auch
  bekannt unter der Bezeichnung \emph{Nebenfach}). Nur wenn ihr
  Medieninformatik, Kognitionswissenschaft, Medizininformatik oder Bioinformatik studierst, braucht ihr euch 
  kein Schwerpunktmodul auszusuchen, da der Anwendungsschwerpunkt in diesen Studiengängen 
  bereits fest integriert ist.
  % (aber trotzdem zur Anmeldung + Info-Veranstaltung gehen!).


Bei der Wahl des Schwerpunktmoduls steht euch so ziemlich alles offen, was die
  Universität Tübingen an Fächern zu bieten hat. Die meisten
  von euch werden sich allerdings für einen der Klassiker -- WiWi,
  Physik, Mathe, \dots -- entscheiden. Vorteil ist:
  Diese Fächer sind
  bereits am Laufen, d. h. konkrete Prüfungsordnungen und -erfahrungen
  existieren, und ihr könnt abschätzen, auf was ihr euch
  einlasst.  Alles, was ihr zu den gängigen Schwerpunktmodulen wissen müsst,
  steht im \emph{Studienplan zum Studium der Informatik an der Universität
  Tübingen}, den ihr bei den weiter hinten aufgeführten offiziellen
  Studienberatern der Fakultät für Informatik erhaltet.

Bei den exotischeren Schwerpunktmodulen seid ihr unter Umständen der erste Studi in
  der über 500-jährigen Geschichte der Universität, der diese
  Fächerkombination "`ausprobiert"'.  Was am Ende -- insbesondere
  später im Studium -- auf euch zukommt, kann sehr überraschend
  sein.  Ihr solltet in jedem Fall auch die Einzelheiten für die
  höheren Semester jetzt schon möglichst verbindlich abklären.  Wenn der
  Studiengang, den ihr als Schwerpunktmodul wählen wollt, nicht allzu
  überlaufen ist, sind die zuständigen Profs, Sekretariate, Assis,
  \dots meist sehr kooperativ; oft sind sie neugierig, wie so ein echter
  Informatiker eigentlich aussieht und was er so isst (oft wollen die
  auch einfach nur fähige Computer-HiWis).  Ihr dürft nicht vergessen,
  das Ausgehandelte auch mit der Informatik abzuklären,
  denn manchmal sind die Anforderungen anderer Fakultäten zu "`lasch"',
  und ihr müsst mehr machen als z.B. ein Geisteswissenschaftler für
  nötig halten würde.

Welches Schwerpunktmodul für euch das richtige ist, müsst ihr letztendlich
  selbst wissen.  Eure Entscheidung sollte aber gründlich überdacht
  sein.
%, denn "`Schwerpunktmodul"' kommt nicht von "`nebenher"' studieren, sondern
%  so manche Semester -- und so manche Semesterferien -- müssen bis zum
%  Bachelor allein in das Schwerpunktmodul investiert werden.


Zusätzlich stellt ihr die ersten Weichen für euren späteren
  beruf\/lichen Einstieg -- und auch wenn ihr es jetzt noch nicht hören
  wollt: Irgendwann steht jeder Studi vor der Frage, wie die Zeit
  zwischen Studium und Rente denn nun am besten zu überbrücken ist.

Macht es euch also nicht allzu einfach, d. h.

\begin{itemize}

\item nehmt nicht nur deswegen - - - - - -, weil ihr dort so wenig für machen müsst,

\item geht nicht nur deswegen zu den - - - - - - - -, nur weil dort
   angeblich die Frauenquote bei 90\% liegen soll

\item und nehmt nicht nur deswegen WiWi, nur weil ihr Geld so toll findet.

\end{itemize}

%\enlargethispage{25pt}
Das Schwerpunktmodul ist die Chance, mal etwas ganz anderes zu machen.  Nix
  Kombjuder, sondern richtiges Leben und sogar oft rumlabern.

Bei der Wahl eures Schwerpunktmodul müsst ihr euch schneller entscheiden. Das Schwerpunktmodul muss dem Prüfungssekretariat bis zum Anfang des 2. Semesters mitgeteilt werden. 

Wichtig ist auch, dass ihr euch \textbf{schnell entscheidet}, denn die
  Fristen, um sich zu Klausuren, Seminaren, Exkursionen, Praktika,
  \dots anzumelden, laufen in manchen Fächern schon Ende Oktober aus.
  Wer zu spät kommt, den bestraft die Uni (und zwar u. U. mit
  Wartezeiten bis zum nächsten Wintersemester!).  Also wer sich
  entschieden hat, muss sich sofort um die nötigen Anmeldungen
  kümmern.  Die Infos dazu erhaltet ihr zumeist von schwarzen Brettern
  bei den jeweiligen Prüfungsämtern bzw. Sekretariaten.

\textbf{Achtung: Die nachfolgenden Artikel ersetzen nicht die
  Informationen zu den Schwerpunktmodulen von offizieller Seite!  Die Artikel
  sollen euch nur Entscheidungshilfe sein und allgemeine Tipps aus Sicht
  der Studenten geben;  Die Auswahl erfolgte willkürlich, d. h. es sind
  nur die Schwerpunktmodule beschrieben, zu denen Informationen von
  Studenten zu erhalten waren (zur Zeit können wir bei den meisten Schwerpunktmodulen keine Angaben zu den zu besuchenden Veranstaltungen machen, wir haben euch aber zu jedem der Schwerpunktmodul den zuständigen Prof. angegeben). Habt ihr euch für ein Schwerpunktmodul
  entschieden, dann besorgt euch unbedingt von den entsprechenden
  Stellen die \emph{aktuellen} Prüfungspläne, Prüfungsordnungen,
  Vorlesungsverzeichnisse, \dots Falls ihr Änderungen
  feststellt, meldet euch doch bitte kurz bei uns, damit wir die Infos
  aktualisieren können: \email{fsi@fsi.uni-tuebingen.de}}

Wenn ihr die zusätzlichen Vorlesungen aus dem Schwerpunktmodul in euren
  Stundenplan aufnehmt, werdet ihr feststellen, dass es, da die
  verschiedenen Fakultäten die Vorlesungszeiten kaum aufeinander
  abstimmen, zu Überschneidungen kommen kann.  In einem solchen Fall
  bleibt euch nichts anderes übrig, als euch für \textbf{eine} der
  Vorlesungen zu
  entscheiden.  Am besten wäre es, ihr würdet euch zu Teams
  zusammenschließen, so dass z.B. einer die Vorlesung bei den WiWis
  mitschreibt, während die anderen brav in der Mathevorlesung sitzen.

\pagebreak


Falls ihr zu einem bestimmten Nebenfach den Studienfachberater sucht
  und Zugang zum WWW habt, schaut auf die Homepage der entsprechenden Fakultät (dort sind die Angaben meist aktuell).
