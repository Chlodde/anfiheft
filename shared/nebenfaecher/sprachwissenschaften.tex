% nebenfaecher/sprachwissenschaften.tex
% kontrolliert: September 2006

Wenn ihr Sprachwissenschaften als Schwerpunktmodul(Nebenfach?) nehmt, könnt ihr zwischen den drei Gebieten Phonetik/Phonologie, Syntax und Semantik einen Bereich heraus wählen.
In dem gewählten Gebiet habt ihr dann die jeweiligen Grundlagenmodule zu besuchen, die jeweils aus einem Proseminar mit einer begleitenden Übung bestehen. 

Bei der Wahl von Sprachwissenschaften als Nebenfach kriegt ihr die Grundbegriffe und Methoden der Sprachwissenschaften vermittelt, so wie Fähigkeiten natürliche Sprachen mit den formalen Sprachkonzepten der Informatik zu vergleichen.

Darüber hinaus gibt es noch einen offiziellen Studienberater im SfS.
  Momentan ist das Dale Gerdemann (Tel. 29-74967).\\
  Zu empfehlen sind auch die Schwarzen Bretter:\\
  Die umfangreichsten Informationen hängen im SfS selbst, aber das
  Wesentliche (z.B. Vorlesungstermine, Seminare etc.) findet ihr auch im Brecht-Bau (Neuphilologikum).
