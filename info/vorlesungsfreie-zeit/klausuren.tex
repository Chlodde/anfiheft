\coronabox{In der Coronazeit sind die Klausuren etwas anders als im
Normalbetrieb. \\ 
z.B. durften wir in den Mensaräumlichkeiten schreiben um Abstand zu wahren. Wie
das bei euch aussieht wird rechtzeitig angekündigt. \\
(rechtzeitig heißt ein Monat vor der Klausur)}

Die größte Freude der vorlesungsfreien Zeit stellen die Klausuren dar. In ihnen
müsst ihr zeigen, was ihr das Semester über gelernt habt. Die Klausuren in
Tübingen finden zumeist in der letzten Woche der Vorlesungszeit und der ersten
Woche der vorlesungsfreien Zeit statt. Solltet ihr durch eine der Klausuren
fallen oder euch entscheiden, dass ihr für eine Klausur noch nicht genug
gelernt habt, finden Nachholtermine zu den Klausuren in den letzten beiden
Wochen der vorlesungsfreien Zeit statt.

Diese Regelung hat den Vorteil, dass ihr in den Semesterferien nur selten
Veranstaltungen habt (Die große Ausnahme bilden hier Praktika).

Wichtig bei den Klausuren ist, dass ihr euch rechtzeitig dafür anmeldet. Die
genauen Anmeldemodalitäten werden euch normalerweise in den ersten Vorlesungen
erläutert. Im Zweifel ist es ratsam, seinem Dozierenden rechtzeitig nochmal zu
fragen. In der Regel gilt jedoch einmal auf Alma anmelden und dann auf der 
Kursplattform der Veranstaltung.
