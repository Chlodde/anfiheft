%%%%%%%%%%%%%%%%%%%%%%%%%%%%%%%%%%%%%%%%%%%%%%%%%%%%%%%%%%%%%%%%%%%%
%%  Diese Datei wird nicht mehr genutzt, aktuell ist stattdessen  %%
%%  anfiheft/info/tuebingen/verkehr.tex                           %%
%%%%%%%%%%%%%%%%%%%%%%%%%%%%%%%%%%%%%%%%%%%%%%%%%%%%%%%%%%%%%%%%%%%%


Es ist sehr sinnvoll, sich ein Semesterticket zu besorgen.  Man zahlt
  zu Beginn des Semesters \semesterticketpreis (der Rest ist im Sozialbeitrag
  enthalten, den ihr an das Studierendenwerk entrichten müsst).  Damit
  kann man dann innerhalb des \emph{Naldo-Bereichs}, das sind die
  Landkreise Tübingen, Reutlingen, Zollern-Alb sowie Sigmaringen
  (und einzelne Strecken in angrenzenden Landkreisen), alle
  öffentlichen Verkehrsmittel benutzen. Dazu zählen sowohl die
  Stadtbusse aller genannten Städte, als auch sämtliche Züge (außer ICE, IC und EC) und
  Regionalbusse, die in dieser Region fahren.

Eine Alternative ist das JugendticketBW für \jugendticketbwpreis pro Semester,
  das in ganz Baden-Württemberg gilt (nicht im Fernverkehr, also ICE, IC, EC und Fernbus).
  Im Gegensatz zum Semesterticket gibt es beim JugendticketBW eine Altersbeschränkung,
  es kann nur von Studierenden gekauft werden, die zu Beginn des Semesters noch 26 Jahre alt oder jünger sind.

Beide Tickets sind in den Reisezentren der DB, der Touristeninformation Tübingens an der Neckarbrücke, und
  online\footnote{\url{https://tickets.naldo.de/}} erhältlich.
  Für den Online-Kauf benötigt ihr eine Login-ID der Uni (beginnend mit \texttt{zx}). Falls ihr eines
  dieser Tickets offline kaufen möchtet, müsst ihr euren Studiausweis sowie die "`Naldo-Bescheinigung"'\footnote{alma > Mein Studium > Studienservice > Bescheide/Bescheinigungen}
  vorzeigen. Für das JugendticketBW müsst ihr als Altersnachweis zusätzlich einen amtlichen Lichtbildausweis zeigen. \\
  Außerdem gilt die Freizeitregelung von Naldo: Hier könnt ihr unter der Woche ab 19 Uhr und am Wochenende ganztägig
  Busse und Bahnen innerhalb des Naldo-Gebiets kostenlos nutzen. Dafür braucht ihr einfach nur euren Studiausweis, auf
  dem sich das Naldo-Logo befindet.
  Besonders interessant, falls die Eltern zu Besuch in Tübingen sind: Samstags kann man im Stadtgebiet (Tarifstufe 11) kostenlos Bus fahren, egal ob man studiert oder nicht.

Es existiert außerdem die Möglichkeit zu anderen regionalen Verkehrsverbünden Anschlusstickets zu kaufen.
Beispielsweise für die Region Stuttgart (VVS-Tarif), allerdings ist das relativ teuer.

Es gibt mehrere Züge zwischen hier und Stuttgart, einen
  Regionalexpress (RE), der stündlich fährt aber recht viele
  Zwischenstopps einlegt und ca. 1 Stunde unterwegs ist sowie einen
  Interregio-Express (IRE), der im 2-Stunden-Takt mit nur einem Halt
  in Reutlingen fährt und nur ca. 45 Minuten braucht.
  Der Fahrpreis beträgt 8,10 EUR, wenn man ein Semesterticket
  besitzt und die Fahrkarte erst ab Bempflingen löst (Bis dahin gilt euer Semesterticket).
  Diese Fahrkarten bekommt ihr an den DB-Automaten unter dem Punkt "`Fahrscheine zur Weiterfahrt"'.
Eine Alternative Lösung ist es entweder mit der 826 Buslinie für 4 Euro bis Stuttgart-Echterdingen (Flughafen Shuttle) zu fahren und von dort weiterzufahren, oder die RB bis Herrenberg zu nehmen und von dort mit der S-Bahn nach Stuttgart zu starten.


Ein nahtloser Übergang vom Naldo zum VVS ist in Bempflingen
  möglich, da aber die Züge nach Stuttgart in Bempflingen nicht halten und in
  Tübingen keine VVS-Tickets erhältlich sind, ist diese Möglichkeit nur für
  Inhaber von VVS-Zeitkarten hilfreich. Eine Alternative mit Mehrfahrtenkarte
  (4 Fahrten; Ticket in Herrenberg kaufen) des VVS und dem Semesterticket
  ist hier mit der Ammertalbahn
  über Herrenberg möglich, die fährt werktags im 30-Minuten Takt und
  bindet neben der S-Bahn nach Stuttgart noch die Orte Pfäffingen, Entringen
  und Altingen an Tübingen an. Für Leute aus Stuttgart und Umgebung besteht die Möglichkeit ein VVS - Ticket als Aufpreis für das Naldo Gebiet dazu zu kaufen.
 Empfehlenswert, falls öfters Fahrten in das VVS Gebiet geplant sind, allerdings kein besonders billiges Vergnügen.

Weitere Zugverbindungen gibt es nach Horb über Rottenburg
  und nach Sigmaringen über Hechingen und Albstadt/Ebingen (Regionalverkehr und IRE).

Fernbusse wie "`Flixbus"', "`DeinBus"' u.A. starten und halten am Tübinger Hauptbahnhof. Weitere Informationen findet ihr auf der Homepage
des jeweiligen Anbieters.

Zum ÖPNV (TüBus) in der Stadt: Die Stadtbusse verkehren i.A. auf
  den meisten Linien werktags von 6 Uhr bis ca. 20 Uhr im 15 bzw. 30 Minuten-Takt,
  außer Linie 5, welche vom Bahnhof über die Morgenstelle zum WHO fährt, im 10
  Minuten Takt. Sehr oft kann man verschiedene Buslinien für die selbe Strecke
  verwenden, die über unterschiedliche Routen zum selben Ziel verkehren.
  Wenn man von der Morgenstelle zum Sand möchte (passiert oft), dann
  fährt man von der Morgenstelle mit der Linie 5
  zum WHO (oder in die Stadt)und dann mit der Linie 2 weiter zum Sand.

  Wollt ihr aus der Stadt auf den Sand zur Informatik fahren, müsst ihr in
  den 2er in Richtung Sand/WHO einsteigen.
  Um von der Stadt (oder vom Bahnhof) zur Morgenstelle zu gelangen kann man
  außer dem 5er auch die Linien 13, 18, 19 oder (mit etwas Umweg) 17 verwenden.
  Aus Richtung Herrenberg kommt man auch mit der Linie 14 über Tübingen West
  auf die Morgenstelle. Der Bus wartet aber leider nicht immer auf die verspätete Bahn!


Wer einmal sehr spät noch irgendwo hin muss (für Leute besonders
  wichtig, die in den ausgelagerten Vororten von Tübingen, wie Hirschau,
  Weilheim, Unterjesingen, ..., wohnen, weil dorthin evtl. nach 20:00 Uhr
  kein Bus mehr fährt), kann sich ein SAM (\textbf{S}ammel\textbf{a}nruf
  \textbf{M}ietfahrzeug) rufen. (Tel: 34000).  Das bringt euch von Haustür
  evtl. über Umwege zu Haustür (in der Innenstadt nur von/zu bestimmten
  SAM-Treffpunkten.

Sehr attraktiv sind auch die Donnerstag- (Clubhaus!), Freitag- und
  Samstag Nacht bis ca. 3 Uhr verkehrende Nachtbusse.  Die Takte sind
  hier zwar nur (halb-)stündlich und die Fahrt dauert vielleicht etwas
  länger als normal, aber Mensch kommt sicher Heim (egal, in
  welchem Zustand).  Also noch ein Grund, das Semesterticket zu
  erwerben, da es auch für die Nachtbusse gilt.

%TODO Hier Pagebreaken
Wer trotz dieses guten Angebots an Nahverkehrsmittel mit dem Auto kommen will/muss,
  sollte folgendes beachten: Kostenlose Parkplätze gibt es nur auf dem Sand,
  nicht auf der Morgenstelle.  Dort benötigt man für den großen Parkplatz eine Nutzungsberechtigung, die man aber nur erhält,
  wenn die Fahrzeit mit Bus und Bahn länger als 40 Minuten dauern würde (Inwieweit es hier für Informatikstudenten Ausnahmen
  gibt, ist derzeit unklar), außerdem ist die Anzahl an Parkplätzen aufgrund von Bauarbeiten sehr begrenzt.
  Erledigen könnt ihr die Freischaltung in dem Büro
  schräg gegenüber dem Studententerminal im Hörsaalzentrum (direkt neben dem Ausgang Richtung Neubau Chemie).
  Ihr müsst dafür aber unbedingt das Datenkontrollblatt mitbringen,
  welches euch per Post zugeschickt wurde.
  Für alle anderen mit kürzerer Fahrzeit steht ein
  kostenpflichtiges Parkhaus zur Verfügung. Wenn ihr eine Parkkarte für das
  ganze Semester kauft, wird's deutlich günstiger.
  \vspace*{3cm}

