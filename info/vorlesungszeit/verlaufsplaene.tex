In höheren Semestern wird von euch erwartet werden, euren Stundenplan entsprechend eurer Prüfungsordnung und eures Modulhandbuches selbst zusammenzusuchen.
Dies bedeutet oft mehrere Stunden mit geöffnetem ALMA, PDF-Dateien sowie Stift und Zettel. Um euch bereits jetzt einen Überblick zu bieten,
was euch nach dem ersten Semester erwartet, haben wir für euch hier die so genannten Studienverlaufspläne zusammengefasst. Diese Pläne sind eine \textbf{Empfehlung}, wie ihr euer Studium gestalten könnt. Ihr müsst euch keinesfalls streng an diese Reihenfolge halten! \medskip
\\
\textbf{Für alle} Informatiker gilt: Ihr müsst eine je nach eurem Studiengang eine unterschiedliche Anzahl an Punkten aus dem \textbf{Studium Professionale} leisten. Oft ist in Prüfungsordnungen und Modulhandbüchern auch von \emph{überfachliche[n] berufsfeldorientierte[n] Kompetenzen} oder \emph{Schlüsselqualifikationen} die Rede. Offiziell geht es hier um das Programm der Uni, in der Praxis kann jedoch alles gewählt werden, das eine Prüfungleistung beinhaltet, für welche man einen benoteten Schein bekommt.\footnote{Abgesehen von Sportkursen.} Oft lohnt es sich auch, sich aus der eigenen comfort zone herauszubewegen und z.B. ein Seminar in den Geisteswissenschaften zu besuchen. \\Außerdem beinhalten alle Studiengänge das \textbf{Teamprojekt}, dies ist ein von einem Lehrstuhl angebotenes Programmier- bzw. Forschungsprojekt, welches ihr in kleineren Gruppen absolvieren müsst. Hier sammelt ihr erste Erfahrungen im teambasierten Arbeiten, setzt euch mit Versionierungssystemen etc. auseinander und erhaltet einen Einblick in die agile Softwareentwicklung.
\vfill \pagebreak 
\subsection*{Informatik}
	%\begin{table}[htbp]
%	\resizebox{\textwidth}{!}{
%		\begin{tabular}{|cccccc|}
%			\hline
%			1. Semester                            & 2. Semester                           & 3. Semester                           & 4. Semester                        & 5. Semester                           & 6. Semester    \\ \hline
%			\multicolumn{1}{|c|}{}                 & \multicolumn{1}{c|}{}                 & \multicolumn{1}{c|}{}                 & \multicolumn{1}{c|}{Mathematik IV} & \multicolumn{1}{c|}{WPF Praktische}   &                \\
%			\multicolumn{1}{|c|}{Mathematik I}     & \multicolumn{1}{c|}{Mathematik II}    & \multicolumn{1}{c|}{Mathematik III}   & \multicolumn{1}{c|}{}              & \multicolumn{1}{c|}{Informatik}       &                \\ \cline{4-5}
%			\multicolumn{1}{|c|}{}                 & \multicolumn{1}{c|}{}                 & \multicolumn{1}{c|}{}                 & \multicolumn{1}{c|}{Theoretische}  & \multicolumn{1}{c|}{WPF Theoretische} & Wahlpflicht    \\ \cline{1-3}
%			\multicolumn{1}{|c|}{Praktische}       & \multicolumn{1}{c|}{Praktische}       & \multicolumn{1}{c|}{}                 & \multicolumn{1}{c|}{Informatik}    & \multicolumn{1}{c|}{Informatik}       & Informatik B   \\ \cline{5-5}
%			\multicolumn{1}{|c|}{Informatik I}     & \multicolumn{1}{c|}{Informatik II}    & \multicolumn{1}{c|}{Algorithmen}      & \multicolumn{1}{c|}{}              & \multicolumn{1}{c|}{WPF Technische}   &                \\ \cline{4-4} \cline{6-6} 
%			\multicolumn{1}{|c|}{}                 & \multicolumn{1}{c|}{}                 & \multicolumn{1}{c|}{}                 & \multicolumn{1}{c|}{}              & \multicolumn{1}{c|}{Informatik}       &                \\ \cline{1-3} \cline{5-5}
%			\multicolumn{1}{|c|}{Technische}       & \multicolumn{1}{c|}{Technische}       & \multicolumn{1}{c|}{Praktikum}        & \multicolumn{1}{c|}{Teamprojekt}   & \multicolumn{1}{c|}{Wahlpflicht}      &                \\
%			\multicolumn{1}{|c|}{Informatik I}     & \multicolumn{1}{c|}{Informatik II}    & \multicolumn{1}{c|}{Tech. Informatik} & \multicolumn{1}{c|}{}              & \multicolumn{1}{c|}{Informatik A}     & Bachelorarbeit \\ \cline{1-5}
%			\multicolumn{1}{|c|}{Studium}      	   & \multicolumn{1}{c|}{Logik}            & \multicolumn{1}{c|}{Schwerpunkt}      & \multicolumn{1}{c|}{Schwerpunkt}   & \multicolumn{1}{c|}{Schwerpunkt}      &                \\ \cline{2-2} \cline{5-5}
%			\multicolumn{1}{|c|}{Professionale}    & \multicolumn{1}{c|}{Proseminar}       & \multicolumn{1}{c|}{}                 & \multicolumn{1}{c|}{}              & \multicolumn{1}{c|}{}					&                \\ \hline
%			30 LP                                  & 30 LP                                 & 30 LP                                 & 30 LP                              & 30 LP                                 & 30 LP          \\ \hline
%		\end{tabular}}
%\end{table}

% PO 2021
\begin{table}[htbp]
\resizebox{\textwidth}{!}{
\begin{tabular}{|cccccc|}
\hline
1.Semester                         & 2. Semester                         & 3. Semester                         & 4. Semester                              & 5. Semester                      & 6. Semester    \\ \hline
\multicolumn{1}{|c|}{Praktische}   & \multicolumn{1}{c|}{Praktische}     & \multicolumn{1}{c|}{Theoretische}   & \multicolumn{1}{c|}{Theoretische}        & \multicolumn{1}{c|}{Proseminar}  & übK            \\ \cline{5-5}
\multicolumn{1}{|c|}{Informatik I} & \multicolumn{1}{c|}{Informatik II}  & \multicolumn{1}{c|}{Informatik I}   & \multicolumn{1}{c|}{Informatik II}       & \multicolumn{1}{c|}{Wahlpflicht} &                \\ \cline{6-6}
\multicolumn{1}{|c|}{}             & \multicolumn{1}{c|}{}               & \multicolumn{1}{c|}{}               & \multicolumn{1}{c|}{}                    & \multicolumn{1}{c|}{Prakt. Inf.} & Wahlpflicht    \\ \cline{1-5}
\multicolumn{1}{|c|}{}             & \multicolumn{1}{c|}{}               & \multicolumn{1}{c|}{}               & \multicolumn{1}{c|}{Mathematik IV}       & \multicolumn{1}{c|}{Wahlpflicht} & Informatik     \\ \cline{6-6}
\multicolumn{1}{|c|}{Mathematik I} & \multicolumn{1}{c|}{Mathematik II}  & \multicolumn{1}{c|}{Mathematik III} & \multicolumn{1}{c|}{}                    & \multicolumn{1}{c|}{Theo. Inf.}  &                \\ \cline{4-5}
\multicolumn{1}{|c|}{}             & \multicolumn{1}{c|}{}               & \multicolumn{1}{c|}{}               & \multicolumn{1}{c|}{Praktische}          & \multicolumn{1}{c|}{übK}         &                \\ \cline{1-3}
\multicolumn{1}{|c|}{Technische}   & \multicolumn{1}{c|}{Technische}     & \multicolumn{1}{c|}{Praktische}     & \multicolumn{1}{c|}{Informatik IV:}      & \multicolumn{1}{c|}{}            & Bachelorarbeit \\ \cline{5-5}
\multicolumn{1}{|c|}{Informatik I} & \multicolumn{1}{c|}{Informatik II}  & \multicolumn{1}{c|}{Informatik III} & \multicolumn{1}{c|}{Teamprojekt}         & \multicolumn{1}{c|}{Wahlpflicht} &                \\ \cline{1-1} \cline{3-4}
\multicolumn{1}{|c|}{übK}          & \multicolumn{1}{c|}{}               & \multicolumn{1}{c|}{Wahlpflicht}    & \multicolumn{1}{c|}{Grundlagen d.}       & \multicolumn{1}{c|}{Informatik}  &                \\ \cline{2-2} \cline{6-6}
\multicolumn{1}{|c|}{}             & \multicolumn{1}{c|}{Technische}     & \multicolumn{1}{c|}{Techn. Inf.}    & \multicolumn{1}{c|}{Machinellen Lernens} & \multicolumn{1}{c|}{}            &                \\ \cline{1-1} \cline{3-5}
\multicolumn{1}{|c|}{}             & \multicolumn{1}{c|}{Informatik III} & \multicolumn{1}{c|}{}               & \multicolumn{1}{c|}{}                    & \multicolumn{1}{c|}{}            &                \\ \hline
30 ECTS                            & 33 ECTS                             & 30 ECTS                             & 30 ECTS                                  & 30 ECTS                          & 27 ECTS        \\ \hline
\end{tabular}}
\end{table}

Abgesehen vom Standardprogramm (Informatik I-III, Mathematik I-III) ist es im Studienplan der Informatik bereits im ersten Semester vorgesehen, Punkte aus dem Bereich des Studium Professionale zu sammeln.  Das Schwerpunktfach (auch Nebenfach genannt) ist ab dem dritten Semester vorgesehen, ihr solltet euch also im zweiten Semester darum kümmern, was ihr als Schwerpunkt wählen wollt und welche Leistungen ihr hierfür erbringen müsst.
\subsection*{Informatik Lehramt}
	% Please add the following required packages to your document preamble:
% \usepackage[normalem]{ulem}
% \useunder{\uline}{\ul}{}
% \begin{table}[htbp]
%	\resizebox{\textwidth}{!}{
%		\begin{tabular}{|ccccccc|}
%			\hline
%			Fachsemester             & LP                      & \multicolumn{2}{c}{Pflicht}                                                                                                                                                                                         & Wahlpflicht                                                                              & Fachdidaktik                                                                          & Bachelorarbeit                                                  \\ \hline
%			\multicolumn{1}{|c|}{1.} & \multicolumn{1}{c|}{15} & \multicolumn{1}{c|}{\begin{tabular}[c]{@{}c@{}}Informatik I \\ (9 LP)\end{tabular}}              & \multicolumn{1}{c|}{\begin{tabular}[c]{@{}c@{}}Einführung in die\\ Technische\\ Informatik\\ (6 LP)\end{tabular}} & \multicolumn{1}{c|}{}                                                                    & \multicolumn{1}{c|}{}                                                                 &                                                                 \\ \cline{1-4} \cline{6-6}
%			\multicolumn{1}{|c|}{2.} & \multicolumn{1}{c|}{12} & \multicolumn{2}{c|}{\begin{tabular}[c]{@{}c@{}}Informatik II\\ (9 LP)\end{tabular}}                                                                                                                                 & \multicolumn{1}{c|}{}                                                                    & \multicolumn{1}{c|}{\begin{tabular}[c]{@{}c@{}}Fachdidaktik I\\ (3 LP)\end{tabular}}  &                                                                 \\ \cline{1-4} \cline{6-6}
%			\multicolumn{1}{|c|}{3.} & \multicolumn{1}{c|}{15} & \multicolumn{2}{c|}{\begin{tabular}[c]{@{}c@{}}Mathematik I /\\ Ausgleichsmodul Mathematik\\ (9 LP)\end{tabular}}                                                                                                   & \multicolumn{1}{c|}{}                                                                    & \multicolumn{1}{c|}{\begin{tabular}[c]{@{}c@{}}Fachdidaktik II\\ (6 LP)\end{tabular}} &                                                                 \\ \cline{1-4} \cline{6-6}
%			\multicolumn{1}{|c|}{4.} & \multicolumn{1}{c|}{15} & \multicolumn{1}{c|}{\begin{tabular}[c]{@{}c@{}}Theoretische \\ Informatik\\ (9 LP)\end{tabular}} & \multicolumn{1}{c|}{\begin{tabular}[c]{@{}c@{}}Informatik\\ der Systeme\\ (6 LP)\end{tabular}}                    & \multicolumn{1}{c|}{}                                                                    & \multicolumn{1}{c|}{}                                                                 &                                                                 \\ \cline{1-5}
%			\multicolumn{1}{|c|}{5.} & \multicolumn{1}{c|}{12} & \multicolumn{2}{c|}{\begin{tabular}[c]{@{}c@{}}Algorithmen\\ (9 LP)\end{tabular}}                                                                                                                                   & \multicolumn{1}{c|}{\begin{tabular}[c]{@{}c@{}}Wahlpflichtmodul I\\ (3 LP)\end{tabular}} & \multicolumn{1}{c|}{}                                                                 &                                                                 \\ \cline{1-5} \cline{7-7} 
%			\multicolumn{1}{|c|}{6.} & \multicolumn{1}{c|}{12} & \multicolumn{2}{c|}{\begin{tabular}[c]{@{}c@{}}Teamprojekt\\ (9 LP)\end{tabular}}                                                                                                                                   & \multicolumn{1}{c|}{Wahlpflichtmodul}                                                    & \multicolumn{1}{c|}{}                                                                 & \begin{tabular}[c]{@{}c@{}}Bachelorarbeit\\ (6 LP)\end{tabular} \\ \hline
%		\end{tabular}}
%\end{table}

% Please add the following required packages to your document preamble:
% \usepackage[normalem]{ulem}
% \useunder{\uline}{\ul}{}
\begin{table}[htbp]
\resizebox{\textwidth}{!}{
\begin{tabular}{|c|c|c|c|c|c|c|}
\hline
FS & ECTS & \multicolumn{2}{c|}{Pflicht}                    & Wahlpflicht   & Fachdidaktik    & Bachelorarbeit \\ \hline
1  & 15   & Praktische              & Technische            &               &                 &                \\
   &      & Informatik I (9 LP)     & Informatik I (6 LP)   &               &                 &                \\ \hline
2  & 12   & \multicolumn{2}{c|}{Praktische Informatik II}   &               & Fachdidaktik I  &                \\
   &      & \multicolumn{2}{c|}{(9 LP)}                     &               & (3 LP)          &                \\ \hline
3  & 15   & Mathematik I /          & Praktische            &               &                 &                \\
   &      & Ausgleichsmodul (9 LP)  & Informatik III (6 LP) &               &                 &                \\ \hline
4  & 15   & \multicolumn{2}{c|}{Theoretische Informatik II} &               & Fachdidaktik II &                \\
   &      & \multicolumn{2}{c|}{(9 LP)}                     &               & (6 LP)          &                \\ \hline
5  & 12   & \multicolumn{2}{c|}{Theoretische Informatik I}  & Wahlpflicht I &                 &                \\
   &      & \multicolumn{2}{c|}{(9 LP)}                     & (3 LP)        &                 &                \\ \hline
6  & 12+6 & \multicolumn{2}{c|}{Technische Informatik II}   & Wahlpflicht I &                 & Bachelorarbeit \\
   &      & \multicolumn{2}{c|}{(9 LP)}                     & (3 LP)        &                 & (6 LP)         \\ \hline
\end{tabular}}
\end{table}

\pagebreak 
Der Stundenplan in Informatik für Lehramt kann recht flexibel belegt werden, da die Module unabhängig voneinander sind. Wer im Zweitfach Mathematik gewählt hat, darf kein Mathe 1 hören, muss aber dafür ein Ausgleichsmodul hören. Als Ausgleichsmodul kann man jedes Wahlpflichtmodul mit gleicher Anzahl an ETCS-Punkten wählen, welches in Alma angeboten wird. Für welche mit einer anderen Kombination als Mathe, ist das Ausgleichsmodul nicht relevant. Die Fachdidaktik-Veranstaltungen finden meist nicht zu den vorgesehen Zeiten im Studienplan statt, da sie teilnehmerabhängig sind. Stattdessen wird mit einer Umfragemail ein Termin gesucht. Die Antwort des Dozenten kann auch mal ein paar Monate auf sich warten lassen.

\subsection*{Bioinformatik}
	\begin{table}[htbp]
	\resizebox{\textwidth}{!}{		
		\begin{tabular}{cccccc}
			\hline
			\multicolumn{1}{|c}{1. Semester}    & 2. Semester                               & 3. Semester                           & 4. Semester                         & 5. Semester                            & \multicolumn{1}{c|}{6. Semester}          \\ \hline
			\multicolumn{1}{|c|}{}              & \multicolumn{1}{c|}{}                     & \multicolumn{1}{c|}{}                 & \multicolumn{1}{c|}{Theoretische}   & \multicolumn{1}{c|}{WPF Praktische}    & \multicolumn{1}{c|}{}                     \\
			\multicolumn{1}{|c|}{Informatik I}  & \multicolumn{1}{c|}{Informatik II}        & \multicolumn{1}{c|}{Algorithmen}      & \multicolumn{1}{c|}{Informatik}     & \multicolumn{1}{c|}{Informatik}        & \multicolumn{1}{c|}{}                     \\ \cline{5-5}
			\multicolumn{1}{|c|}{}              & \multicolumn{1}{c|}{}                     & \multicolumn{1}{c|}{}                 & \multicolumn{1}{c|}{}               & \multicolumn{1}{c|}{Chemie II}         & \multicolumn{1}{c|}{Bachelorarbeit}       \\ \cline{1-4}
			\multicolumn{1}{|c|}{}              & \multicolumn{1}{c|}{}                     & \multicolumn{1}{c|}{}                 & \multicolumn{1}{c|}{Stochastik}     & \multicolumn{1}{c|}{}                  & \multicolumn{1}{c|}{}                     \\ \cline{5-5}
			\multicolumn{1}{|c|}{Mathmematik I} & \multicolumn{1}{c|}{Mathematik II}        & \multicolumn{1}{c|}{Mathmematik III}  & \multicolumn{1}{c|}{}               & \multicolumn{1}{c|}{WPF Lebenswissen-} & \multicolumn{1}{c|}{}                     \\ \cline{4-4} \cline{6-6} 
			\multicolumn{1}{|c|}{}              & \multicolumn{1}{c|}{}                     & \multicolumn{1}{c|}{}                 & \multicolumn{1}{c|}{}               & \multicolumn{1}{c|}{schaften}          & \multicolumn{1}{c|}{Wahlpflicht}          \\ \cline{1-3} \cline{5-5}
			\multicolumn{1}{|c|}{Biomoleküle}  & \multicolumn{1}{c|}{Chemie I}             & \multicolumn{1}{c|}{Molekulare}       & \multicolumn{1}{c|}{Teamprojekt}    & \multicolumn{1}{c|}{Proseminar}        & \multicolumn{1}{c|}{Bioinformatik}        \\ \cline{5-6} 
			\multicolumn{1}{|c|}{und Zelle}     & \multicolumn{1}{c|}{(Teil B)}             & \multicolumn{1}{c|}{Biologie}         & \multicolumn{1}{c|}{}               & \multicolumn{1}{c|}{Studium}           & \multicolumn{1}{c|}{WPF Bioinfo / Info}   \\ \cline{1-4}
			\multicolumn{1}{|c|}{Chemie I}      & \multicolumn{1}{c|}{Einführung BioInfo.} & \multicolumn{1}{c|}{Neurobiologie}    & \multicolumn{1}{c|}{Grundlagen der} & \multicolumn{1}{c|}{Professionale}     & \multicolumn{1}{c|}{Lebenswissenschaften} \\ \cline{2-2} \cline{6-6} 
			\multicolumn{1}{|c|}{(Teil A)}      & \multicolumn{1}{c|}{}                     & \multicolumn{1}{c|}{}                 & \multicolumn{1}{c|}{Bioinformatik}  & \multicolumn{1}{c|}{}                  &                                           \\ \cline{1-1} \cline{3-3} \cline{5-5}
			                                    & \multicolumn{1}{c|}{}                     & \multicolumn{1}{c|}{Prakt. Neurobio.} & \multicolumn{1}{c|}{}               &                                        &                                           \\ \cline{3-4}
		\end{tabular}}
\end{table}
Das Bioinformatik-Studium beinhaltet in den ersten Semestern neben dem Standardprogramm der Informatik eine gute Portion Chemie und Zellbiologie. Später kommen dann Molekularbiologie (zusammen mit dem Studiengang molekulare Medizin) sowie noch mehr Chemie hinzu.
\pagebreak 
\subsection*{Medieninformatik}
	% \begin{table}[htbp]
% 	\resizebox{\textwidth}{!}{
% 		\begin{tabular}{cccccc}
% 			\hline
% 			\multicolumn{1}{|c}{1. Semester}     & 2. Semester                        & 3. Semester                            & 4. Semester                           & 5. Semester                            & \multicolumn{1}{c|}{6. Semester}      \\ \hline
% 			\multicolumn{1}{|c|}{}               & \multicolumn{1}{c|}{}              & \multicolumn{1}{c|}{}                  & \multicolumn{1}{c|}{Theoretische}     & \multicolumn{1}{c|}{Graphische}        & \multicolumn{1}{c|}{}                 \\
% 			\multicolumn{1}{|c|}{Informatik I}   & \multicolumn{1}{c|}{Informatik II} & \multicolumn{1}{c|}{Algorithmen}       & \multicolumn{1}{c|}{Informatik}       & \multicolumn{1}{c|}{Datenverarbeitung} & \multicolumn{1}{c|}{}                 \\
% 			\multicolumn{1}{|c|}{}               & \multicolumn{1}{c|}{}              & \multicolumn{1}{c|}{}                  & \multicolumn{1}{c|}{}                 & \multicolumn{1}{c|}{}                  & \multicolumn{1}{c|}{Bachelorarbeit}   \\ \cline{1-5}
% 			\multicolumn{1}{|c|}{}               & \multicolumn{1}{c|}{Informatik}    & \multicolumn{1}{c|}{}                  & \multicolumn{1}{c|}{WP Informatik /}  & \multicolumn{1}{c|}{Wahlpflicht}       & \multicolumn{1}{c|}{}                 \\
% 			\multicolumn{1}{|c|}{Mathmematik I}  & \multicolumn{1}{c|}{der Systeme}   & \multicolumn{1}{c|}{Mathmematik III}   & \multicolumn{1}{c|}{Medieninformatik} & \multicolumn{1}{c|}{Informatik /}      & \multicolumn{1}{c|}{}                 \\ \cline{2-2} \cline{4-4} \cline{6-6} 
% 			\multicolumn{1}{|c|}{}               & \multicolumn{1}{c|}{}              & \multicolumn{1}{c|}{}                  & \multicolumn{1}{c|}{WP Medienwiss.}   & \multicolumn{1}{c|}{Medieninformatik}  & \multicolumn{1}{c|}{Wahlpflicht}      \\ \cline{1-1} \cline{3-4}
% 			\multicolumn{1}{|c|}{User Interface} & \multicolumn{1}{c|}{Mathematik II} & \multicolumn{1}{c|}{Grundlagen der}    & \multicolumn{1}{c|}{}                 & \multicolumn{1}{c|}{}                  & \multicolumn{1}{c|}{Informatik /}     \\ \cline{5-5}
% 			\multicolumn{1}{|c|}{Design}         & \multicolumn{1}{c|}{}              & \multicolumn{1}{c|}{Multimediatechnik} & \multicolumn{1}{c|}{Teamprojekt}      & \multicolumn{1}{c|}{WP}                & \multicolumn{1}{c|}{Medieninformatik} \\ \cline{1-3} \cline{6-6} 
% 			\multicolumn{1}{|c|}{Einführung}    & \multicolumn{1}{c|}{Internet-}     & \multicolumn{1}{c|}{Bildverarbeitung}  & \multicolumn{1}{c|}{}                 & \multicolumn{1}{c|}{Medienwiss.}       & \multicolumn{1}{c|}{Studium}          \\ \cline{4-5}
% 			\multicolumn{1}{|c|}{Medienwiss.}    & \multicolumn{1}{c|}{technologien}  & \multicolumn{1}{c|}{}                  & \multicolumn{1}{c|}{Proseminar}       & \multicolumn{1}{c|}{Stud. Profess.}    & \multicolumn{1}{c|}{Professionale}    \\ \hline
% 			                                     &                                    &                                        &                                       &                                        &                                       
% 		\end{tabular}}
% \end{table}

% PO 2021
\begin{table}[htbp]
\resizebox{\textwidth}{!}{
\begin{tabular}{|cccccc|}
\hline
1.Semester                         & 2. Semester                        & 3. Semester                            & 4. Semester                           & 5. Semester                            & 6. Semester     \\ \hline
\multicolumn{1}{|c|}{Praktische}   & \multicolumn{1}{c|}{Praktische}    & \multicolumn{1}{c|}{Theoretische}      & \multicolumn{1}{c|}{Theoretische}     & \multicolumn{1}{c|}{Bildverarbeitung}  & Wahlpflicht     \\
\multicolumn{1}{|c|}{Informatik I} & \multicolumn{1}{c|}{Informatik II} & \multicolumn{1}{c|}{Informatik I}      & \multicolumn{1}{c|}{Informatik II}    & \multicolumn{1}{c|}{}                  & Medieninfo      \\ \cline{5-6} 
\multicolumn{1}{|c|}{}             & \multicolumn{1}{c|}{}              & \multicolumn{1}{c|}{}                  & \multicolumn{1}{c|}{}                 & \multicolumn{1}{c|}{Graphische}        & übK             \\ \cline{1-4}
\multicolumn{1}{|c|}{}             & \multicolumn{1}{c|}{}              & \multicolumn{1}{c|}{}                  & \multicolumn{1}{c|}{Praktische}       & \multicolumn{1}{c|}{Datenverarbeitung} &                 \\ \cline{6-6} 
\multicolumn{1}{|c|}{Mathematik I} & \multicolumn{1}{c|}{Mathematik II} & \multicolumn{1}{c|}{Mathematik III}    & \multicolumn{1}{c|}{Informatik IV:}   & \multicolumn{1}{c|}{}                  &                 \\ \cline{5-5}
\multicolumn{1}{|c|}{}             & \multicolumn{1}{c|}{}              & \multicolumn{1}{c|}{}                  & \multicolumn{1}{c|}{Teamprojekt}      & \multicolumn{1}{c|}{Wahlpflicht}       &                 \\ \cline{1-4}
\multicolumn{1}{|c|}{User}         & \multicolumn{1}{c|}{Technische}    & \multicolumn{1}{c|}{Praktische}        & \multicolumn{1}{c|}{Wahlpflicht}      & \multicolumn{1}{c|}{Medieninfo}        & Bachelorarbeit \\ \cline{5-5}
\multicolumn{1}{|c|}{Experience}   & \multicolumn{1}{c|}{Informatik II} & \multicolumn{1}{c|}{Informatik III}    & \multicolumn{1}{c|}{Informatik}       & \multicolumn{1}{c|}{Wahlpflicht}       &                 \\ \cline{1-1} \cline{3-3}
\multicolumn{1}{|c|}{Wahlpflicht}  & \multicolumn{1}{c|}{}              & \multicolumn{1}{c|}{Grundlagen d.}     & \multicolumn{1}{c|}{}                 & \multicolumn{1}{c|}{Medienwiss.}       &                 \\ \cline{2-2} \cline{4-6} 
\multicolumn{1}{|c|}{Medienwiss.}  & \multicolumn{1}{c|}{Internet-}     & \multicolumn{1}{c|}{Multimediatechnik} & \multicolumn{1}{c|}{Ethik Proseminar} & \multicolumn{1}{c|}{Proseminar}        &                 \\ \cline{1-1} \cline{3-5}
\multicolumn{1}{|c|}{}             & \multicolumn{1}{c|}{technologien}  & \multicolumn{1}{c|}{}                  & \multicolumn{1}{c|}{}                 & \multicolumn{1}{c|}{}                  &                 \\ \hline
30 ECTS                            & 33 ECTS                            & 30 ECTS                                & 30 ECTS                               & 30 ECTS                                & 27 ECTS         \\ \hline
\end{tabular}}
\end{table}

Als Medieninformatiker müsst ihr das Standardprogramm der Informatik absolvieren, außerdem macht ihr in den ersten Semestern Ausflüge in die Medienwissenschaften und belegt Kurse in der Webgestaltung und -Programmierung.
\subsection*{Medizininformatik}
	% \begin{table}[htbp]
% 	\resizebox{\textwidth}{!}{
% 		\begin{tabular}{cc|c|ccc}
% 			\hline
% 			\multicolumn{1}{|c|}{1. Semester}                      & 2. Semester      & 3. Semester       & \multicolumn{1}{c|}{4. Semester}      & \multicolumn{1}{c|}{5. Semester}        & \multicolumn{1}{c|}{6. Semester}    \\ \hline
% 			\multicolumn{1}{|c|}{}                                 &                  & Physik I          & \multicolumn{1}{c|}{Physik II}        & \multicolumn{1}{c|}{Wahlpflicht}        & \multicolumn{1}{c|}{}               \\
% 			\multicolumn{1}{|c|}{Informatik I}                     & Informatik II    &                   & \multicolumn{1}{c|}{}                 & \multicolumn{1}{c|}{Informatik}         & \multicolumn{1}{c|}{}               \\ \cline{3-5}
% 			\multicolumn{1}{|c|}{}                                 &                  & User Interface    & \multicolumn{1}{c|}{Grundlagen}       & \multicolumn{1}{c|}{Wahlpflicht Bio-/}  & \multicolumn{1}{c|}{Bachelorarbeit} \\ \cline{1-2}
% 			\multicolumn{1}{|c|}{}                                 & Internet-        & Design            & \multicolumn{1}{c|}{Bioinformatik}    & \multicolumn{1}{c|}{Medizininformatik}  & \multicolumn{1}{c|}{}               \\ \cline{3-5}
% 			\multicolumn{1}{|c|}{Mathematik I}                     & technologien     & Telemedizin       & \multicolumn{1}{c|}{}                 & \multicolumn{1}{c|}{Medzinische}        & \multicolumn{1}{c|}{}               \\ \cline{2-2} \cline{6-6} 
% 			\multicolumn{1}{|c|}{}                                 &                  &                   & \multicolumn{1}{c|}{Humanbiologie IV} & \multicolumn{1}{c|}{Visualisierung}     & \multicolumn{1}{c|}{Wahlpflicht}    \\ \cline{1-1} \cline{5-5}
% 			\multicolumn{1}{|c|}{Grundlagen der Medizininformatik} & Mathematik II    & Ökonomie i.d.    & \multicolumn{1}{c|}{}                 & \multicolumn{1}{c|}{Wahlpflicht}        & \multicolumn{1}{c|}{Informatik}     \\ \cline{4-4} \cline{6-6} 
% 			\multicolumn{1}{|c|}{}                                 &                  & Medizininformatik & \multicolumn{1}{c|}{}                 & \multicolumn{1}{c|}{Medizin / Biologie} & \multicolumn{1}{c|}{Studium}        \\ \cline{1-3} \cline{5-5}
% 			\multicolumn{1}{|c|}{Humanbiologie I}                  & Humanbiologie II & Humanbiologie III & \multicolumn{1}{c|}{Teamprojekt}      & \multicolumn{1}{c|}{Studium}            & \multicolumn{1}{c|}{Professionale}  \\ \cline{1-1} \cline{6-6} 
% 			\multicolumn{1}{|c|}{Med. Terminlogie}                 &                  &                   & \multicolumn{1}{c|}{}                 & \multicolumn{1}{c|}{Professionale}      &                                     \\ \cline{1-5}
% 			                                                       &                  & Biostatistik      &                                       &                                         &                                     \\ \cline{3-3}
% 		\end{tabular}}
% \end{table}

% PO 2021
\begin{table}[htbp]
\resizebox{\textwidth}{!}{
\begin{tabular}{|cccccc|}
\hline
1.Semester                              & 2. Semester                           & 3. Semester                            & 4. Semester                           & 5. Semester                           & 6. Semester     \\ \hline
\multicolumn{1}{|c|}{Praktische}        & \multicolumn{1}{c|}{Praktische}       & \multicolumn{1}{c|}{User}              & \multicolumn{1}{c|}{Grundlagen}       & \multicolumn{1}{c|}{Wahlpflicht}      & Wahlpflicht     \\
\multicolumn{1}{|c|}{Informatik I}      & \multicolumn{1}{c|}{Informatik II}    & \multicolumn{1}{c|}{Experience}        & \multicolumn{1}{c|}{Bioinformatik}    & \multicolumn{1}{c|}{Informatik}       & Bioinfo         \\ \cline{3-3} \cline{5-6} 
\multicolumn{1}{|c|}{}                  & \multicolumn{1}{c|}{}                 & \multicolumn{1}{c|}{Praktische}        & \multicolumn{1}{c|}{}                 & \multicolumn{1}{c|}{Medizinische}     & Wahlpflicht     \\ \cline{1-2} \cline{4-4}
\multicolumn{1}{|c|}{}                  & \multicolumn{1}{c|}{}                 & \multicolumn{1}{c|}{Informatik III}    & \multicolumn{1}{c|}{Physik II}        & \multicolumn{1}{c|}{Visualisierung}   & Medizininfo     \\ \cline{3-3} \cline{5-6} 
\multicolumn{1}{|c|}{Mathematik I}      & \multicolumn{1}{c|}{Mathematik II}    & \multicolumn{1}{c|}{Physik I}          & \multicolumn{1}{c|}{}                 & \multicolumn{1}{c|}{Telemedizin}      &                 \\ \cline{4-4}
\multicolumn{1}{|c|}{}                  & \multicolumn{1}{c|}{}                 & \multicolumn{1}{c|}{}                  & \multicolumn{1}{c|}{Humanbiologie IV} & \multicolumn{1}{c|}{}                 &                 \\ \cline{1-3} \cline{5-5}
\multicolumn{1}{|c|}{Grundlagen d.}     & \multicolumn{1}{c|}{Einf. Internet-}  & \multicolumn{1}{c|}{Humanbiologie III} & \multicolumn{1}{c|}{}                 & \multicolumn{1}{c|}{Wahlplicht}       & Bachelorarbeit  \\ \cline{4-4}
\multicolumn{1}{|c|}{Medizininfo.}      & \multicolumn{1}{c|}{technologien}     & \multicolumn{1}{c|}{}                  & \multicolumn{1}{c|}{}                 & \multicolumn{1}{c|}{Medizin/Biologie} &                 \\ \cline{1-3} \cline{5-5}
\multicolumn{1}{|c|}{Humanbiologie I}   & \multicolumn{1}{c|}{Humanbiologie II} & \multicolumn{1}{c|}{Biostatistik}      & \multicolumn{1}{c|}{Teamprojekt}      & \multicolumn{1}{c|}{Proseminar}       &                 \\ \cline{1-1} \cline{3-3} \cline{5-6} 
\multicolumn{1}{|c|}{Med. Terminologie} & \multicolumn{1}{c|}{}                 & \multicolumn{1}{c|}{Ethik (übK)}       & \multicolumn{1}{c|}{}                 & \multicolumn{1}{c|}{übK}              & übK             \\ \hline
\multicolumn{1}{|c|}{}                  & \multicolumn{1}{c|}{}                 & \multicolumn{1}{c|}{}                  & \multicolumn{1}{c|}{}                 & \multicolumn{1}{c|}{}                 &                 \\ \hline
30 ECTS                                 & 30 ECTS                               & 30 ECTS                                & 30 ECTS                               & 30 ECTS                               & 30 ECTS         \\ \hline
\end{tabular}}
\end{table}

Im Gegensatz zu den restlichen Informatikern müssen die Medizininformatiker lediglich Mathe I-II sowie Informatik I-II besuchen, da ein Großteil der Punkte in den ersten 4 Semestern für Humanbiologie (zusammen mit dem Studiengang Medizintechnik) und Physik abfällt. Eine weitere Sonderregelung: Ihr könnt im 4. Semester wählen, ob ihr Algorithmen oder Grundlagen der Bioinformatik hören wollt. Wollt ihr Algorithmen hören, tauscht ihr das am besten mit dem Studium Professionale, so dass ihr dieses im vierten Semester hört und Algorithmen im fünften.

\linkbig{https://www.wsi.uni-tuebingen.de/studium/studiengaenge.html}{Informationen zu Studiengängen}{Weitere Infos zum Studienverlauf sowie \\Prüfungsordnungen und Modulhandbücher findet ihr auf der Webseite des WSI}
\pagebreak