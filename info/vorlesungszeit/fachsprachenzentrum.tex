Am Fachsprachenzentrum\footnote{\url{https://www.uni-tuebingen.de/fsz/}} finden semesterbegleitende Kurse und während der Semesterferien zwei- bis vierwöchige Intensivkurse in verschiedenen Sprachen statt.\\	%TODO insert \link{}{}?
Neben den Einsteiger- und Anfängerkursen (UNIcert I/II mit je drei Niveaustufen) kann man auch fachspezifische Kurse belegen z. B. Beruf und Studium, Naturwissenschaften, Wirtschaftswissenschaften, etc.  Ob man direkt mit einer höheren Niveaustufe beginnen darf (oder im Grundkurs beginnen muss) entscheidet ein Einstufungstest vor Ort.  Gegeben werden die Kurse meist von Muttersprachlern und haben mittleres bis hohes Niveau.
Die Einschreibung zu den semesterbegleitenden Kursen starten häufig schon vor dem Semester. Die genauen Zeiten werden im Internet (s. o.) bekannt gegeben.  Gerade bei begehrten Kursen (d. h. fast alle Einsteigerkurse wie z. B. Spanisch I oder Italienisch I) sollte man aber bei der  Einschreibung schnell sein. Meist sind alle Plätze schon am ersten Einschreibetag vergeben.\\
Die Kurse sind für Studenten kostenlos. Wer sich angemeldet hat muss jedoch auch erscheinen, da Anwesenheitspflicht besteht. Ansonsten droht eine Sperre für weitere Kurse am FSZ für das kommende Semester in der jeweiligen Sprache.
  
Eine Alternative zu alle dem ist natürlich die klassische Volkshochschule.  Ihr Angebot kann im Internet\footnote{\url{https://www.vhs-tuebingen.de/}} eingesehen werden.  Im Angebot der vhs finden sich Kurse vieler Sprachen, von Englisch bis Hindi oder Suaheli in jeweils mehreren Niveaustufen und mit vielen Spezialkursen wie Konversationskurse, Grammatikkurse, Reisekurse, Kinderkurse, etc.  Sie kosten zwischen 90 und 150 EUR (je nach Sprache, Kursart, Dauer, ...) - zumeist sind es jedoch etwa 90 EUR. Das Niveau dieser Kurse ist im Allgemeinen relativ niedrig, allerdings kann man das auch vorher telefonisch bei der vhs erfragen und sich dort beraten lassen.	%TODO insert \link{}{}?

\vfill
