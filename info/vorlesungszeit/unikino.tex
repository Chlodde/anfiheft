Das Unikino ist eine von Studenten geleitete Veranstaltungsreihe, bei der z.B. unbekannte und in Vergessenheit geratene, aber auch recht aktuelle Filme gezeigt werden, oft auch in englischer Originalfassung. Ihr zahlt ihr pro Semester einmalig \EUR{0,30}, eine Vorstellung kostet \EUR{2} Eintritt (übrigens gelten diese Preise auch für Nicht-Studierende!). Alle Filme werden im Hörsaal 24 im Kupferbau gezeigt (Haltestelle Hölderlinstraße, dort direkt an der großen Kreuzung).\\
In den letzten Vorlesungswochen vor der Weihnachtspause wird der Film "` Die Feu\-er\-zang\-en\-bow\-le"' gezeigt. Dieser Film ist nicht nur Kult, die Fachschaft Informatik schenkt bei dieser Vorstellung auch traditionell Glühwein aus. Vorbeischauen lohnt sich also doppelt! Man sollte jedoch bei dieser Vorstellung früh da sein, der Hörsaal ist meistens rappelvoll.

\subsection*{Spielplan}
%Der Standardspieltag ist Dienstag, Filme mit normaler Länge beginnen um 19:45 Uhr, Filme mit Überlänge bereits um 19:15. Beim Wunschfilmabend wählt das Publikum vor der Vorstellung, welcher Film aus einer Auswahl von 5 Filmen gezeigt wird. Der Eintritt pro Film beträgt 2 Euro.  
%\medskip \\
\renewcommand{\arraystretch}{1.2}
\begin{tabular}{l l}
\textbf{Datum} & \textbf{Filmtitel}     			\\
15.10. & Bohemian Rhapsody							\\
22.10. & Free Solo									\\
29.10. & We the Coyotes (franz. Filmtage)			\\
05.11. & The Hate U Give							\\
12.11. & Us											\\
19.11. & Burning									\\
26.11. & Before the Flood (Fridays for Future)		\\
03.12. & Die Feuerzangenbowle						\\
10.12. & Anna and the Apocalypse					\\
07.01. & Shazam!									\\
14.01. & If Beale Street Could Talk					\\
21.01. & Once Upon a Time in Hollywood				\\
28.01. & Shoplifters								\\
04.02. & Instant Family								\\

\end{tabular}

\textbf{Filmbeginn}: 19:45 Uhr, bei Überlänge 19:15 Uhr\\
\textbf{Eintritt}: 2 \euro ~(+0,30 \euro beim ersten Mal)\\
\textbf{Ort}: Hörsaal 24 im Kupferbau\\
Alle Filme in Originalsprache mit Untertiteln\\

Das Programm ist wegen Zeitplanung zu 98 Prozent sicher.
Alle weiteren Infos zum Unikino findet ihr auf \url{https://www.unifilm.de/studentenkinos/tuebingen} oder auch auf Facebook.