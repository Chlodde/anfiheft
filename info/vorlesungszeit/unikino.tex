Das Unikino ist eine von Studierenden geleitete Veranstaltungsreihe, bei der
unbekannte und in Vergessenheit geratene, aber auch recht aktuelle Filme
gezeigt werden. Dabei werden die Filme meistens in Originalfassung (mit
Untertiteln) gezeigt. Eine Vorstellung kostet \EUR{2} Eintritt, außerdem zahlt
ihr pro Semester einmalig \EUR{0,30} (übrigens gelten diese Preise auch für
Nicht-Studierende!). Alle Filme werden im Hörsaal 24 im Kupferbau gezeigt
(Haltestelle Hölderlinstraße, dort direkt an der großen Kreuzung).\\ 
In den letzten Vorlesungswochen vor der Weihnachtspause wird der Film 
"` Die Feu\-er\-zang\-en\-bow\-le"' gezeigt. Dieser Film ist nicht nur Kult,
auch die Vorstellung selbst ist ein Erlebnis. Bei diesem Film handelt es sich 
um Mitmachkino. Mehr soll hier aber nicht verraten werden.

% \coronabox{Wie und ob das Unikino im Sommersemster 2022 stattfindet ist zur
% Zeit leider noch nicht klar. Tendenz aber eher nein. Infos dazu und ggf. einen
% Spielplan findest du auf der Webseite des
% Unikinos\footnote{\url{https://www.unifilm.de/studentenkinos/tuebingen}} oder
% auf Facebook/Twitter/Instagram.}

\vspace*{-1em}

\subsection*{Spielplan}
Der Standardspieltag ist Dienstag, Filme mit normaler Länge beginnen um 19:45 Uhr, Filme mit Überlänge bereits um 19:15. Beim Wunschfilmabend wählt das Publikum vor der Vorstellung, welcher Film aus einer Auswahl von 5 Filmen gezeigt wird. Der Eintritt pro Film beträgt 2 Euro.  
\medskip \\
\renewcommand{\arraystretch}{1.2}
\begin{tabular}{l l}
\textbf{Datum} & \textbf{Filmtitel}     			\\
18.10. & TBA \\
25.10. & TBA \\
01.11. & TBA \\
08.11. & TBA \\
15.11. & TBA \\
22.11. & TBA \\
29.11. & TBA \\
06.12. & TBA \\
13.12. & TBA \\
20.12. & TBA \\
10.01. & TBA \\
17.01. & TBA \\
24.01. & TBA \\
31.01. & TBA \\

\end{tabular}

\textbf{Filmbeginn}: 19:45 Uhr, bei Überlänge 19:15 Uhr\\
\textbf{Eintritt}: 2 \euro ~(+0,30 \euro beim ersten Mal)\\
\textbf{Ort}: Hörsaal 24 im Kupferbau\\
Alle Filme in Originalsprache mit Untertiteln

% Das Programm ist wegen Zeitplanung zu 98 Prozent sicher.
Alle weiteren Infos zum Unikino findet ihr auf \url{https://www.unifilm.de/studentenkinos/tuebingen} oder auch auf Facebook.
