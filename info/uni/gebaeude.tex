\textbf{Morgenstelle}\\
Auf der Morgenstelle befindet sich das Hörsaalzentrum, das Gebäude der Universität, welches sowohl die meisten als auch die größten Hörsäle beherbergt. Hier werdet ihr unter anderem auch die meisten eurer Grundvorlesungen haben.\\
Die Morgenstelle erreicht man am bequemsten mit dem Bus. In 98\% der Fälle solltet ihr an der Haltestelle \emph{BG Unfallklinik} aussteigen. Die anderen Haltestellen \emph{Auf der Morgenstelle} und \emph{Botanischer Garten} eignen sich nur, wenn ihr z.B. in N10 oder N11 müsst.\\ Bei den Hochhäusern auf der Morgenstelle gilt: Jedes Gebäude besitzt einen Buchstaben, der in mehr oder weniger gut sichtbarem Orange außen am Gebäude angebracht ist.  Die Raumnummern orientieren sich an den Gebäudebuchstaben und Stockwerken, z.B. befindet sich der Raum C9A03 im C-Bau im 9. Stockwerk.\footnote{Im Gegensatz dazu ist der Raum 7E02 NICHT im E-Bau im 7. Stockwerk. Achtet auf die Reihenfolge!} Wenn ihr Gebäude C und D betretet, befindet ihr euch bereits im dritten Stockwerk. Warum man sich das so ausgedacht hat, wissen wir auch nicht. Wichtig sind im C-Bau vor allem die Poolräume, welche sich im zweiten Stockwerk links vom Aufzug befinden. \medskip \\
%TODO Orientierung Morgenstelle
Einen Lageplan der wichtigen Gebäude und wichtigsten Orte auf der Morgenstelle findet ihr im Anhang.

\textbf{Sand}\\
Das Wilhelm-Schickard-Institut für Informatik oder auch einfach nur Sand genannt\footnote{Basierend auf den Postadressen Sand 1, 6, 7, 8, 13 und 14} ist das Institut der Informatik in Tübingen. \\ In den ersten Semestern werdet ihr hier höchstens euer Prüfungssekretariat und/oder die Fachschaft besuchen. Auch wenn das Institut etwas außerhalb der restlichen Universität liegt, besticht es dennoch durch sein großzügiges Platzangebot und zahlreiche Möglichkeiten zur Freizeitgestaltung in Pausen. Unter anderem befinden sich auf dem Gelände ein Volleyballfeld, ein Fußballfeld und eine Tischtennisplatte. Bälle und Tischtennisschläger liegen in unserem Fachschaftsraum und dürfen von jedem Studierenden der Informatik ausgeliehen werden.\\
Da es auf dem Sand anfangs immer schwierig ist, sich bei den Raumbezeichnungen zurechtzufinden, haben wir einen kleinen Übersichtsplan pro Stockwerk erstellt. Ihr findet diesen im Anhang.
%TODO Orientierung Sand

\textbf{TTR\footnote{aka: Die Machine von Learning-Straße}}\\
Beginnend mit dem Tübingen AI Research Building wird für die Informatik auf der Oberen Viehweide mehr Platz geschaffen. Wer also Professoren für Machine Learning und Computergraphik sucht wird dort fündig.

\textbf{Uni Tal}\\
Ein großer, wenn auch für die Informatik eher irrelevanter Teil, der Universität liegt an der Wilhelmstraße dem sogenannten Tal. Als Informatiker hat man hier höchstens Nebenfachveranstaltungen.
\vfill 