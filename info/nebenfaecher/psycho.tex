% nebenfaecher/psycho.tex

In diesem Schwerpunktmodul werdet ihr in die Gebiete der Psychologie eingeführt.
Hier müsst ihr einmal die Vorlesung "`Forschungsmethoden der Psychologie"' hören, ausserdem jeweils eine Vorlesung aus den beiden nachfolgenden Gebieten "`Grundlagen"' und "`Anwendung"'.
Es gibt zum Beginn des Wintersemesters immer eine Einführungsveranstaltung für Nebenfachstudierende, der Termin stand zum Drucktermin leider noch nicht fest.
Wichtig: Manche Vorlesungen werden im Wintersemester angeboten (z.B. Forschungsmethoden der Psychologie),
andere dagegen im Sommersemester. Genauere Informationen zu den Veranstaltungen findet ihr im Campus-System.
Ansprechpartner bei den Psychologen ist Prof. Dr. Barbara Kaup.

\pagebreak

Pflicht:
\begin{itemize}
\item Forschungsmethoden der Psychologie 4LP
\end{itemize}


Grundlagen:
\begin{itemize}
 \item VL Biologische Psychologie 6LP
 \item VL Sozialpsychologie 6LP
 \item VL Persoenlichkeitspsychologie 6 LP
 \item VL Allgemeinte Psychologie A -- Wahrnehmung 3 LP
 \item VL Allgemeinte Psychologie B -- Lernen, Emotion und Motivation 3LP
 \item VL Allgemeinte Psychologie C -- Aufmerksamkeit und Denken 3LP
 \item VL Allgemeine Psychologie D -- Gedaechtnis und Sprache 3 LP
 \item VL Entwicklungspsychologie -- Teil 1 3LP
 \item VL Entwicklungspsychologie -- Teil 2 3LP (aufbauend -- Teil 1 muss positiv absolviert sein)
\end{itemize}


Anwendung:
\begin{itemize}
\item VL Wissens-, Kommunikations- und Medienpsychologie 6 LP
\item VL Wirtschaftspsychologie 6LP
\item VL Klinische Psychologie 6LP
\end{itemize}




Ansprechpartner aus der Informatik ist Prof. Schilling\footnote{\email{schilling@uni-tuebingen.de}}.

Aktuelle Informationen oder Änderungen findet ihr im Web\footnote{
  \url{http://www.pi.uni-tuebingen.de/studium.html}}

  Die Psychologie als Nebenfach sollte
  nicht unterschaetzt werden. Entgegen der verbreiteten Meinung wird hier
  nicht nur "`gelabert"', es handelt sich um eine empirische Wissenschaft
  und man muss schon einiges lernen.
  Und da das Nebenfach bei Informatikern selten ist, wird man schnell zum
  "`Einzelkämpfer"'.

Im Vergleich zu den Kognitionswissenschaftlern (Kognis), habt ihr damit mehr Auswahlmöglichkeiten innerhalb der Veranstaltungen der Psychologie. Kognis hören außerdem Philosophie, Linguistik und Biologie, was bei euch nicht geplant ist.

