"`Medizin als Nebenfach?"', "`Ja, was hat das denn überhaupt mit Informatik
  zu tun?"'. Diese und ähnliche Fragen werdet ihr, falls ihr
  Euch für Medizin als Nebenfach
  entscheidet, des öfteren zu hören bekommen.
  Die Vorlesungen, die ihr im Grundstudium hört, sind
  nämlich zu 99\% für Mediziner, Biologen oder Biochemiker gedacht; und die
  haben natürlich mit Computern wenig bis gar nichts am Hut. Ihr 
  lernt also \textbf{nicht}, wie man beispielsweise einen Computertomographen
  programmiert oder ein Krankenhausinformationssystem aufbaut, sondern ihr seid
  damit beschäftigt, euch erstmal medizinisches Basiswissen und viele, viele
  Fachbegriffe anzueignen. 


Auch die Medizin lässt sich hier in Tübingen als Schwerpunktmodul nehmen. Dabei kriegt ihr die Grundkenntnisse der medizinischen Terminologie, so wie die Grundlagen der Physiologie und Physik vermittelt. Dabei belegt ihr die zwei Module "`Medizinische Grundlagen I"' und "`Medizinische Grundlagen II"'. In der Medizin werdet ihr viele Teilprüfungen ablegen. Aus dem arithmetischen Mittel der Einzelprüfungen errechnet sich dann eure Endnote. "`Medizinische Grundlagen I"` besteht aus den Pflichtveranstaltungen "`Medizinische Terminologie"' und "`Anatomie"'. "`Medizinische Grundlagen II"' besteht aus "`Physiologie"' und "`Medizinische Physik"'.

Weitere Veranstaltungen, Hinweise etc. sind auf der Nebenfach-Homepage von
Dr. Lautenbacher aufgeführt. Die Veranstaltungen werden meistens kurz vor
Vorlesungbeginn bekanntgegeben:
\url{http://www.medizin.uni-tuebingen.de/nfmi/nf_plan.htm}


Für Nebenfächler in der Medizin zuständig ist:

Dr. med. H. Lautenbacher: \\
Institut für Medizinische Informationsverarbeitung \\
Westbahnhofstraße 55 \\
Tel.: 07071/29-85070 \\
\url{http://www.medizin.uni-tuebingen.de/nfmi/nf_index.htm}


\textbf{Wichtig}: Bevor ihr irgendeine Mediziner-Veranstaltung besucht, bei Herrn Lautenbacher melden.
