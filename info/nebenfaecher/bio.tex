Fällt eure Wahl auf Biologie als Schwerpunktmodul belegt ihr die zwei Teilmodule "`Biomoleküle und Zelle"' (BMZ) und "`Zellbiologie, Genetik und Mikrobiologie"'. 

BMZ ist eine einführende Vorlesungsreihe die aus den drei Gebieten Zellbiologie, Genetik und Mikrobiologie die Basics vermittelt. Zusätzlich zu der Vorlesung erwartet euch noch ein Praktikum was parallel zu der Vorlesung läuft.

Um die Klausur antreten zu können müsst ihr zuvor noch das Praktikum erfolgreich absolviert haben.

Beim Modul Zellbiologie/Mikrobiologie/Genetik müsst ihr zwei der drei Fächer wählen. Zellbiologie und Genetik gehören zur Vorlesung "`Molekulare Biologie I"'. Darin werden auf diverse Themen wie Metabolismus, Zelldifferenzierung, Signalübertragung Zelltod etc. von Zellen eingegangen. Im Genetik Teil von Molekulare Biologie I werdet ihr euch mit der Organisation von Genen im Genom so wie diverse Mechanismen der Genregulation kennen lernen. Im Vordergrund stehen dabei auch die Vermittlung der Methoden. Die Vorlesung Molekulare Biologie II setzt sich mit der Genetik und dem Metabolismus von Mikroorganismen auseinander.


Im Vergleich zum Bioinformatikstudium hat man als Informatiker mit Biologie als Schwerpunktmodul einen viel weniger umfangreichen Stundenplan, da man als "`Bio-Nebenfächler"' tatsächlich nur Biologie hört (die Bioinformatiker haben im Grundstudium einiges an Chemieveranstaltungen zu absolvieren).

Die Gebäude der Biologie befinden sich auf der Morgenstelle: am E-Bau geht man vorbei, wenn man von der Mensa Richtung Hörsaalzentrum geht, die Botanik in Richtung Botanischer Garten und das Verfügungsgebäude zwischen D-Bau und Botanischem Garten. Die Vorlesungen werden meist im Verfügungsgebäude gehalten, während die Praktika im E-Bau stattfinden. Im dritten Stock des E-Baus hängen die Pläne mit den Lehrveranstaltungen aus.

Wenn ihr Fragen rund um das Nebenfach Biologie habt, könnt ihr euch an die Beratungsstelle wenden:
\email{beratung@biologie.uni-tuebingen.de}




