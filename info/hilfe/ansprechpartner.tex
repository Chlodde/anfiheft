%&\subsection{Studentische Studienberatung}
Wenn ihr Fragen bezüglich eures Studienablaufes, eurer Prüfungsordnung, Orientierungs- oder Zwischenprüfungen etc. habt, kann euch mit hoher
Wahrscheinlichkeit die \textbf{studentische Studienberatung} weiterhelfen. Das sind Studenten des jeweiligen Studiengangs, die mit allen Regularien vertraut sind
und einen guten Draht zu den PA-Vorsitzenden haben. Im Folgenden sind das: \\

%\vfill %\pagebreak
\textbf{Informatik, Bioinformatik} \quad \studBeratungInfo \\
\begin{tabular}{rl}
  Mail: & \email{studienberatung@informatik.uni-tuebingen.de}
\end{tabular}

\textbf{Kognitionswissenschaft} \quad \studBeratungKogni \\
\begin{tabular}{rl}
	Mail: & \email{kogni-beratung@fsi.uni-tuebingen.de}
\end{tabular}

\textbf{Informatik Lehramt} \quad \studBeratungLehramt \\
\begin{tabular}{rl}
  Mail: & \email{lehramt@informatik.uni-tuebingen.de}
\end{tabular}

\textbf{Medieninformatik} \quad \studBeratungMedien \\
\begin{tabular}{rl}
  Mail: & \email{medieninformatik@uni-tuebingen.de}
\end{tabular}

\textbf{Medizininformatik} \quad \studBeratungMedizin \\
\begin{tabular}{rl}
  Mail: & \email{medizininformatik@uni-tuebingen.de}
\end{tabular}

Die studentischen Studienberater haben keine festen Sprechzeiten oder Büros. Termine können daher nur nach Vereinbarung per E-Mail arrangiert werden.

\subsection{Universitäre Studienberatung}
Neben der studentischen Studienberatung gibt es auch noch eine studiengangspezifische Studienberatung durch MitarbeiterInnen des Fachbereichs Informatik\footnote{\url{https://uni-tuebingen.de/de/74360}}. \\

\pagebreak
