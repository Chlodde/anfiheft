% infra/mailinglisten.tex

Die FSI betreibt einige Mailinglisten, die allen Informatikstudierenden die Möglichkeit 
bieten, Fragen an andere Informatikstudierende zu richten, deren Fragen zu beantworten 
und sonstige Informationen weiterzuleiten. Jeder ist dazu eingeladen die Mails, welche 
an diese Listen gehen zu lesen und selbst an eine Liste zu schreiben. Wir empfehlen euch 
grade für die erste Zeit eine Anmeldung.
%\pagebreak

\newcommand{\mladressen}[1]{
    Anmelden: Leere Mail an {\footnotesize \email{#1-subscribe@fsi.uni-tuebingen.de}} \\
    Abmelden: Leere Mail an {\footnotesize \email{#1-unsubscribe@fsi.uni-tuebingen.de}} \\
    Hilfetext: Mail mit Betreff \emph{help} an {\footnotesize \email{#1-request@fsi.uni-tuebingen.de}}}

\begin{description}

  \item[info-studium\At fsi.uni-tuebingen.de (Studium)] ~\\
  	Auf dieser Mailingliste solltet ihr euch unbedingt anmelden, da das der einzige 
  	Kanal der Profs ist, um mit euch in Verbindung zu treten. Hier geht es um alle 
  	Themen, die euer Studium betreffen, wie kurzfristige Änderungen oder Ankündigungen 
  	zum Semester.

    \mladressen{info-studium}

  \item[info-talk\At fsi.uni-tuebingen.de (Laber)] ~\\
    Diese Liste dient der allgemeinen Kommunikation zwischen Informatikern.
    Hier landen u.a. Diskussionen, die auf einer der anderen Listen begonnen
    haben und dort thematisch nicht mehr passen.

    \mladressen{info-talk}

  \item[info-jobs\At fsi.uni-tuebingen.de (Stellenangebote)] ~\\
    Wer auf der Suche nach einem Job ist, sollte sich auf dieser Verteilerliste
    für Stellenangebote anmelden. Achtung, auf dieser Liste sollten nur die
    Angebote selbst und keine Diskussionen darüber landen.

    \mladressen{info-jobs}

  \item[info-lehramt\At fsi.uni-tuebingen.de (Lehramt)] ~\\
  Diese Liste dient zur Kommunikation zwischen den Lehrämtlern. Hier kann man sich 
  untereinander helfen, wenn Probleme im Studium auftauchen.
  
 \mladressen{info-lehramt}
  
  \item[versuche\At fsi.uni-tuebingen.de (Teilnahme an Versuchen)] ~\\
    Wer an wissenschaftlichen Versuchen und Studien teilnehmen will, findet auf dieser 
    Mailingliste Einladungen zur Teilnahme an Versuchen.
  
  \mladressen{versuche}
    
  \item[sport\At fsi.uni-tuebingen.de (Sport)] ~\\
  	Wer ab und zu Lust dazu hat, auf dem Sand Sport zu machen (Volleyball, Fußball, ...) 
  	findet auf dieser Mailingliste leicht Menschen, die mitspielen wollen, oder erfährt 
  	wann man selbst mitspielen kann.
  	
  	\mladressen{sport}
  	
  \item[coding\At fsi.uni-tuebingen.de (Hackathons uä.)] ~\\
  	Lust mal aus dem Keller raus zu kommen und mit anderen Leuten in einem stickigen Raum 
  	zu sitzen, Pizza zu essen und ein Wochenende damit zu verbringen mit Semikolons um 
  	sich zu werfen? Dann ist diese Liste für dich interessant.
  	
  	\mladressen{coding}

\end{description}

Für die Experten: Mails von diesen Listen zu filtern funktioniert am Besten mit dem Header 'List-Id:'.
