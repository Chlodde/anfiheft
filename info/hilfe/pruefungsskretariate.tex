Euer Prüfungssekretariat ist die erste Anlaufstelle bei Fragen rund ums Thema Prüfungen. Hierzu zählen unter anderem die An- und Abmeldung von Prüfungen, Anrechnung von außerfakultären Veranstaltungen,
das Ausstellen von Bescheinigungen und Zeugnissen und irgendwann\footnote{Dieser Zeitpunkt wird schneller da sein, als euch lieb ist} die Anmeldung der Bachelorarbeit.\\
%Im ersten Semester müsst ihr auf jeden Fall ein Mal auf euer Prüfungssekretariat, um eure Erstanmeldung zu vollziehen, da ihr sonst keine Prüfungen mitschreiben könnt! %Ansonsten müsst ihr das Prüfungssekretariat eigentlich nur aufsuchen, wenn die Prüfungsanmeldung über Alma mal wieder nicht funktioniert.
Die Sekretariate haben manchmal recht abenteuerliche Öffnungszeiten. Um euch die Suche zu ersparen, haben wir hier die verschiedenen Sekretariate zusammen mit ihren Öffnungszeiten zusammengetragen. Eine Übersicht der Prüfungssekretariate gibt es auch auf der Uni-Website \footnote{\url{https://uni-tuebingen.de/de/74384}}. Da es immer wieder mal zu Änderungen, z.B. durch Urlaube, kommen kann, lohnt es sich meistens da mal einen Blick drauf zu werfen, bevor man sich auf den Weg macht.\\
Je nach Studiengang ist ein anderes Prüfungssekretariat für euch zuständig: \\

%\vfill 
%\pagebreak 

\textbf{(Medien-)Informatik B.Sc./M.Sc., Informatik NF}\\
Renate Hallmayer, Raum B118 (Sand)\\
Telefon 07071 - 29 - 78962\\
E-Mail: \email{hallmr@informatik.uni-tuebingen.de}\\
Mo-Fr 10–12 Uhr\\
Di, Do: 14.00–16.00 Uhr\\

\textbf{Medizininformatik B.Sc./M.Sc., Bioinformatik B.Sc./M.Sc.}\\ 
Monika Weber, Raum B316 (Sand)\\
Telefon 07071 - 29 - 78952\\
E-Mail: \email{monika.weber@uni-tuebingen.de}\\
Di, Mi, Do: 09:30–11:30 Uhr\\

\textbf{Informatik-Lehramt}\\
Tanja Mader, Raum 3A22 (E-Bau, Morgenstelle)\\
Telefon 07071 - 29 - 76167\\
E-Mail: \email{pruefungsamt.lehramt@mnf.uni-tuebingen.de}\\
Mo: 11:00-12:00 Uhr (nur telefonisch)\\
Di: 9:00-11:00 Uhr\\

Alle Sekretariate besitzen neben der jeweiligen Tür ein Postfach. Statt die Sprechzeiten in Anspruch zu nehmen, können Formulare daher meistens auch unkompliziert in das entsprechende Postfach eingeworfen werden.\\
