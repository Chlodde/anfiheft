Euer Prüfungssekretariat ist die erste Anlaufstelle bei Fragen rund ums Thema
Prüfungen. Hierzu zählen unter anderem die An- und Abmeldung von Prüfungen,
Anrechnung von außerfakultären Veranstaltungen, das Ausstellen von
Bescheinigungen und Zeugnissen und irgendwann\footnote{Dieser Zeitpunkt wird
schneller da sein, als euch lieb ist} die Anmeldung der Bachelorarbeit.

%Im ersten Semester müsst ihr auf jeden Fall ein Mal auf euer
%Prüfungssekretariat, um eure Erstanmeldung zu vollziehen, da ihr sonst keine
%Prüfungen mitschreiben könnt! %Ansonsten müsst ihr das Prüfungssekretariat
%eigentlich nur aufsuchen, wenn die Prüfungsanmeldung über Alma mal wieder
%nicht funktioniert.

Die Sekretariate haben manchmal recht abenteuerliche Öffnungszeiten. Um euch
die Suche zu ersparen, haben wir hier die verschiedenen Sekretariate zusammen
mit ihren Öffnungszeiten zusammengetragen. Eine Übersicht der
Prüfungssekretariate gibt es auch auf der Uni-Website
\footnote{\url{https://uni-tuebingen.de/de/214849}}. Da es immer wieder mal zu
Änderungen, z.B. durch Urlaube, kommen kann, lohnt es sich meistens da mal
einen Blick drauf zu werfen, bevor man sich auf den Weg macht. Da seht ihr außerdem, ob 
ihr per Mail einen Termin machen sollt.

Je nach Studiengang ist ein anderes Prüfungssekretariat für euch zuständig: \\

%\vfill 
%\pagebreak 

\textbf{Informatik B.Sc.}\\
Manuela Wieder, Wilhelmstraße 19\\
Telefon 07071 - 29 - 77543\\
E-Mail: \email{pruefungsamt.informatik-bachelor@uni-tuebingen.de}\\
Do: 11:00-13:00 Uhr (mit Terminvereinbarung)\\

\textbf{Informatik M.Sc.}\\
Nicole Grauer, Wilhelmstraße 19\\
Telefon 07071 - 29 - 75091\\
E-Mail: \email{pruefungsamt.informatik-master@uni-tuebingen.de}\\
Di, Do: 11:00-12:00 Uhr (mit Terminvereinbarung)\\

\textbf{Bioinformatik, Medizininformatik, Medieninformatik}\\
Susanne Mischorr, Wilhelmstraße 19\\
Telefon 07071 - 29 - 75991\\
E-Mail: \email{pruefungsamt.bioinformatik@uni-tuebingen.de}/ \email{pruefungsamt.medizininformatik@uni-tuebingen.de}/ \email{pruefungsamt.medieninformatik@uni-tuebingen.de} \\
Tel. Sprechzeiten: Mo: 11:00-12:00 Uhr \& Mi: 13:00-14:00 Uhr\\

\textbf{Informatik-Lehramt}\\
Tanja Mader, Wilhelmstraße 19\\
Telefon 07071 - 29 - 76167\\
E-Mail: \email{pruefungsamt.lehramt@mnf.uni-tuebingen.de}\\
Di: 9:00-11:00 Uhr (mit Terminvereinbarung)\\

Alle Sekretariate besitzen neben der jeweiligen Tür ein Postfach. Statt die
Sprechzeiten in Anspruch zu nehmen, können Formulare daher meistens auch
unkompliziert in das entsprechende Postfach eingeworfen werden.\\
