Liebe Erstsemester,

auf den folgenden Seiten haben wir versucht euch das Wichtigste für eine
erste Orientierung im Uni-Alltag zusammenzutragen. Dieses Heft
ist also vor allem eine Sammlung allerlei wichtiger Infos und Tipps\footnote{
Bitte beachtet: Auch wir können uns irren, rechtlich verbindlich ist nur die
jeweilige Studienordnung!}, die ihr
euch sonst selbst mühsam durch viele Fragen und Rennerei %von schwarzen Brettern, die quer über ganz Tübingen
%verteilt sind, oder aus Sekretariaten, deren erstaunliche
%Öffnungszeiten\footnote{Inzwischen haben wir für euch unter folgender Adresse 
%eine inoffizielle Übersicht zusammengestellt:\\
%\url{https://www.fsi.uni-tuebingen.de/studium/sekretariate}}
%auch für den Rest eures Studiums die ganz große Unbekannte bleiben werden,
zusammensuchen müsstet.

%Unsere persönlichen Erfahrungen am WSI -- Wilhelm-Schickard-Institut -- sowie diverse Kontaktmöglichkeiten
%haben wir natürlich auch eingebracht.  %Diese betreffen vor allem die
%Nebenfächer und damit das spannendste Thema eures Studienbeginns.

\coronabox{
%    Wie ihr wahrscheinlich schon im Vorfeld mitbekommen habt, ist	dieses \\
%    Semester vieles anders als sonst. Es wird im Semester immer wieder zu \\
%    Änderungen kommen, und auch für uns ist die Situation neu -- wir sind \\
%    also in vielerlei Hinsicht ebenso planlos wir ihr. \\
%    Es kommt also darauf an die Pandemie so gut es geht zusammen zu meistern, \\
%    scheut euch daher nicht uns mit Fragen zu kontaktieren.\vspace{1em} \\
%    Denn auch wenn wir die Antwort 	vielleicht nicht selbst wissen,\\
%    kennen wir wahrscheinlich die richtige Ansprechperson.\vspace{1em} \\
  Wie ihr vielleicht schon mitbekommen habt, steht auch dieses Semester wieder
  unter dem Thema Corona. Das heißt für euch, dass ihr im Semester spontan auf
  neue Regelungen reagieren dürft. \\

  Aus den letzten Semestern wissen wir, dass es wichtig ist andere Studierende zu
  finden. Alleine kommt man im Studium nicht weit. Gerade in Mathe sind
  Übungsgruppen unerlässlich.

  Das Gute ist, dass ihr tatsächlich zum ersten Mal seit eineinhalb Jahren wieder
  in Uni-Gebäude und Vorlesungsräumen sitzen könnt. (Vorrausgesetzt ihr seid
  geimpft oder genesen) Darum solltet ihr Leute finden können.\\

	Aktuelle Informationen zu Corona-Regelungen findet ihr auf der Webseite\\
  der Uni\footnote{\url{https://uni-tuebingen.de/universitaet/infos-zum-coronavirus/}}
	und auch auf der Webseite der Fachschaft\footnote{\url{https://www.fsi.uni-tuebingen.de/covid19/faq}}.\\
	Ansonsten solltet ihr euch auf jeden Fall um eine stabile Internet-Verbindung, 
	ein funktionierendes Micro und eine Webcam kümmern.}

Ihr haltet bereits die \number\auflage. Auflage unseres Info-Heftes in den
Händen. Trotzdem mag das ein oder andere noch zu verbessern sein. Haltet
euch mit Kritik also nicht zurück, denn jede Anregung für die kommenden
Auf\/lagen ist herzlich willkommen.

Was noch bleibt, ist, euch zur Mitarbeit in der Fachschaft einzuladen.  Die
Fachschaft ist ein "`lockerer Haufen"' Info-Studis, die sich einmal wöchentlich
für eine Fachschaftssitzung treffen. Dort geht es vor allem um das WSI, die 
Mathematisch-Naturwissenschaftliche Fakultät und die Universität Tübingen, 
aber auch um ganz allgemeine Themen. Kommt doch einfach mal mit euren Fragen 
und Anregungen oder einfach mit etwas Neugier auf einem unserer Treffen vorbei.  
Zeit und Ort erfahrt ihr von unserer Webseite oder direkt von uns.

\bigskip

Viel Spaß beim Studieren,

Eure Fachschaft
%\email{fsi@fsi.uni-tuebingen.de}
\vfill
\bigskip

\eject
