
Der Inhalt des "`reinen"' Informatikstudiums muss zu einem großen Teil auch von den Bioinformatikern, 
Medieninformatikern und Kognitionswissenschaftlern gelernt werden. Die Informatiker müssen zusätzlich 
jedoch ein Nebenfach wählen, von denen die gängigsten weiter hinten in diesem Heft und im Anschluss 
an die Begrüßung durch den Dekan am Freitag vor dem Vorlesungsbeginn vorgestellt werden.

Diese Vorlesungen werden eure Zeit (auch abends und am Wochenende) in den ersten vier Semestern beanspruchen: 

\begin{itemize}

\item Die für viele interessantesten Vorlesungen \textbf{Informatik I} und \textbf{Informatik II} sind
  sicherlich auch die zentralsten des Grundstudiums und werden im ersten und zweiten
  Semester gehört. Ihr werdet hier und in den begleitenden
  Übungen in einige Programmiersprachen (welche das sind ist von Professor zu Professor
  leicht unterschiedlich) und die nötigste Theorie eingeführt. Beide Vorlesungen werden dem
  Bereich Praktische Informatik zugeordnet.

\item Eine erste Einführung in die Theoretische Informatik erhaltet ihr im dritten Semester
  mit der Vorlesung \textbf{Informatik III}. Auch sie wird von Übungen begleitet und orientiert
  sich zumeist an dem netten Büchlein "`Theoretische Informatik -- kurzgefasst"' von Uwe Schöning.
  Ihr werdet den Mann und sein wirklich gutes Buch lieben lernen -- und an der Mathematik zweifeln,
  wenn ihr erfahrt, dass es immer wahre arithmetische Formeln geben wird, die man nicht beweisen kann...
  
\item Im vierten Semester lernt ihr die Funktionsweise von \textbf{Algorithmen} (Informatik IV).

\item Wie man einen Computer bastelt wird im Modul \textbf{Technische Informatik} behandelt -- 
  zumindest theoretisch. Hierzu müsst ihr euch "`Einführung in die Technische Informatik"' und "`Informatik der Systeme"' anhören.

\item Unerwähnt darf natürlich auch insbesondere die Mathematik nicht bleiben. So steht für
  euch in den ersten vier Semestern \textbf{Mathematik I bis IV} auf den Programm. Die ersten
  drei Teile führen in die wichtigsten Methoden ein (üblicherweise entlang des Buchs "`Mathematik
  für Informatiker und Bioinformatiker"', welches von zwei unserer Informatikprofs und einem
  Tübinger Mathematiker verfasst wurde), während sich hinter Teil IV für Bioinformatiker die Vorlesung 
  Stochastik, für Kognitionswissenschaftler die Vorlesung Angewandte Mathematik und für Informatiker/Medieninformatiker 
  wahlweise Stochastik oder Numerik verbirgt. 

\item Nur die Informatiker müssen noch das \textbf{TI-Basispraktikum}
     absolvieren, welches die praktische Anwendung der TI-Vorlesung demonstriert.
     In diesem Praktikum gibt es eine Abfrage, deren Nichtbestehen
     mehr Freizeit an diesem Tag zufolge hat.  Allerdings wird nicht jede Gruppe abgefragt,
     sondern immer eine durch Zufall ermittelt (d.h. man kann theoretisch immer abgefragt werden).
     Die Abfrage sollte man nicht unterschätzen (aber auch nicht überschätzen). 
     Es gibt eine Vorbesprechung zu Beginn des Semesters, zu der man auf jeden Fall erscheinen muss!
     Die Anmeldung muss im Allgemeinen bereits am Ende des Vorsemesters erfolgen, auch hier gilt
     es rechtzeitig Ausschau nach einem entsprechenden Aushang zu halten. In welchem Semester ihr
     das hinter euch bringt, steht euch frei.

\item Außerdem muss im vierten Semester noch ein \textbf{Programmierpraktikum} gemeistert werden. Kognitionswissenschaftler sind hiervon befreit. Im Semester werdet Ihr in einer begleitenden Vorlesung Techniken zum Programmieren im Team erlernen. Unterschätzt dieses Praktikum nicht - es wird sehr arbeitsintensiv sein!

\item Ihr solltet zudem rechtzeitig daran denken, dass in Tübingen Veranstaltungen
  aus dem Bereich \textbf{Schlüsselqualifikationen} bis zum Ende des Bachelors absolviert werden müssen. Praktisch wird da fast
  alles akzeptiert, zum Beispiel "`Ethik in den Biowissenschaften"' oder "`Einführung in das Recht"'.  Seminare vom Studium Professionale können ebenfalls teilweise eingebracht werden.
  Auch an unserem Institut werden regelmäßig Vorlesungen wie zB "`Grundlagen wissenschaftlichen Arbeitens"' und "`Scientific Writing"'  angeboten, die als Vorbereitung für Bachelor -  / Masterarbeit zu empfehlen sind.
  Schaut am besten in eurem Studienplan nach, wann dafür Zeit vorgesehen ist und fangt nicht zu spät damit an euch um diese Punkte zu kümmern, 

\end{itemize}

Als Orientierungsprüfung müsst ihr die Veranstaltungen "`Mathematik I"' oder "`Mathematik II"' sowie "`Informatik I"' oder "`Informatik II"' bestehen.
% Fick dich, JP.


Darüber hinaus müsst ihr noch einige Veranstaltungen für den Wahlpflichtbereich hören, das findet
meist zwischen dem 3. und dem 6. Semester statt. Ihr müsst Module für technische, theoretische und praktische Informatik füllen, außerdem ein viertes, in das ihr alles packen könnt. Hier könnt ihr
euch endlich euren Stundenplan selber frei zusammenstellen!


