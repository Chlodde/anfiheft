Auch wenn es manchmal vor lauter Einmischungen des Landes schwer zu erkennen
  ist, verwaltet sich die Universität selbst.  Die Mitbestimmungsmöglichkeiten
  der Studierenden sind jedoch begrenzt, da in jedem wichtigen Gremium die
  Profs per Gesetz die Mehrheit haben.  Dennoch haben wir die Möglichkeit,
  etwas durch gute Argumentation und Engagement zu erreichen. An unserer
  Fakultät ist das meistens auch einvernehmlich mit den Profs möglich.  So
  haben wir zum Beispiel in den letzten Jahren einen Vorschlag zur Änderung der
  Prüfungs- und Studienordnung eingebracht, der letztendlich erfolgreich war.
  Auch auf anderen Gebieten wie zum Beispiel bei den Öffnungszeiten der
  Bibliothek wurden unsere Anregungen aufgenommen. Auf jeden Fall bieten sich
  auf diesem Gebiet viele interessante Möglichkeiten der Mitarbeit, die
  einen Einblick hinter die Kulissen der Universität gewähren und viel Spaß
  machen.

Die \textbf{Fachschaft} (FS) ist aus studentischer Sicht wohl das wichtigste
  Instrument für den Dialog zwischen Profs und Studis.  Formell besteht sie
  aus allen Studenten des Fachbereichs bzw. - juristisch gesehen - sogar der
  Fakultät. Im praktischen Sprachgebrauch besteht die FS aber aus all denjenigen
  Studentinnen und Studenten, die Lust haben, sich etwas für andere zu engagieren
  oder einfach nur an der netten Atmosphäre einer Fachschaft teilhaben wollen.
  Die Fachschaft kümmert sich um fast alle studentischen Belange im Fachbereich,
  gibt Tipps zu Prüfungen, betreut im ersten Semester die neuen Studierenden und
  organisiert natürlich auch Feste.
  Dementsprechend sind eine Vielzahl von wichtigen, unwichtigeren,
  zeitaufwändigen, aber auch einfachen Tätigkeiten zu vergeben.

Letztendlich kann  es in der Fachschaft nie zu viele Engagierte geben. Neue
Gesichter sind immer gern gesehen. Schreibt uns eine Mail, um zu erfahren, wann
im jeweiligen Semester der w\"ochentliche Sitzungstermin ist und schaut einfach
mal rein!

Der \textbf{Fachbereitsrat} ist das erste Gremium auf Fachbereichsebene und hat
de facto keinen nennenswerten Einfluss.

Der \textbf{Fakultätsrat} ist die Versammlung einer gewissen Anzahl von
  Profs, Mitarbeitern und Studenten einer Fakultät, in der alle wichtigen
  fakultätsbezogenen Themen behandelt werden.  Bei uns sind das seit 2010
  nun also nicht nur Informatiker sondern Vertreter aus der gesamten
  Mathematisch- Naturwissenschaftlichen Fakultät.  In dem Gremium sitzen
  neben den fünf Studis\footnote{Jedes Jahr finden im Sommer Uni-Wahlen
  statt, bei denen alle Studierenden aufgerufen sind, ihre Vertretung im
  Fakultätsrat, im Studierendenrat und im Senat zu wählen.} noch drei wissenschaftliche
  Mitarbeiter, drei sonstige Mitarbeiter, fünf Professoren, fünf
  Fachbereichssprecher sowie der Dekan und die Prodekane.

Die \textbf{Studienkommission} ist für die Angelegenheiten des Studiums und der
Lehre zuständig. Hier werden u.a. Studienpläne erstellt und das Lehrangebot
besprochen. Die Studkomm setzt sich aus dem Studiendekan, drei Professoren,
zwei Vertretern des wissenschaftlichen Dienstes und vier Studierenden zusammen.
Gewählt werden die Mitglieder durch den Fakultätsrat. Es existieren Studienkommissionen
f\"ur die Informatik (inklusive Bio-, Medien- und Medizininformatik),
das Lehramt Informatik, die Kognitionswissenschaft.

Ein Gremium, das nur von Studis besetzt wird, ist der \textbf{Studierendenrat} (StuRa).
Der StuRa ersetzt seit Winter 2013 den Allgemeinen Studierendenausschuss (AStA).
Der AStA hat in Baden-W\"urttemberg seit den 70er-Jahren keine wirklichen
Rechte mehr besessen, erst mit dem Regierungswechsel zu Gr\"un-Rot wurde den
Studierenden in Form der \textbf{Verfassten Studierendenschaft} wieder das Recht
gegeben, sich hochschulpolitisch zu äußern. Bis dahin war der AStA
lediglich f\"ur kulturelle und sportliche Belange zust\"andig. Der StuRa wiederum
vertritt nun die Studierenden in s\"amtlichen Hochschulpolitischen Angelegenheiten
und darf selbst Geld verwalten (das einmal im Semester eingezogen wird).

Der StuRa besteht aus verschiedenen Fraktionen, die sich wiederum aus
hochschulpolitischen Gruppen zusammensetzen. F\"ur die Fachschaft am wichtigsten
ist die \textbf{Fachschaften-Vollversammlung} (FSVV), die sich aus Fachschaftlern zusammensetzt und die Belange der Fachschaften vertritt.

Der \textbf{Senat} ist das höchste Gremium der Universität zusammen mit
  dem Hochschulrat.  Neben vielen, vielen Profs\footnote{alle Dekane, ein
  paar gewählte und das Rektorat} sitzen hier 4 gewählte Studenten und
  einige Mitarbeiter.  Durch Mitarbeit in Kommissionen und Anfragen gibt es
  auch in so einem großen Gremium Möglichkeiten, auf studentische Sichtweisen
aufmerksam zu machen.  Im
  Senat werden Berufungen, Studienordnungen, Richtungsentscheidungen und
  eben alles, was irgendwie wichtig ist, verhandelt.  Er setzt zu allen
  möglichen Themen Kommissionen ein, die ihm einen Teil der Arbeit
  abnehmen.


Vollständigkeitshalber sei hier auch noch der \textbf{Hochschulrat}
  genannt, der eine Art Aufsichtsrat ist und in den  Entscheidungen der
  Uni das letzte Wort hat.  Auch hier sind die Studenten mit einem
  Vertreter vertreten, neben Uni-Internen sitzen hier auch Leute aus
  Politik, Wirtschaft und Adel. Durch das neue Landeshochschulgesetz,
von dem ihr sicher schon gehört habt, wird dieses Gremium erheblich
aufgewertet -- damit nimmt der Einfluss von Externen und damit des
Ministerium zu.


Die Kandidaten für die oben genannten Gremien (abgesehen von den
  Fachschaften) kommen meist aus den verschiedenen Hochschulgruppen.  In
  Tübingen gibt es davon 6  aktive: Das eher konservative Spektrum wird vom
"`Ring Christlich Demokratischer Studenten"' (RCDS) abgedeckt,
  %, und der   JU-HG (Junge Union Hochschulgruppe) abgedeckt, die
  %der wie der Name schon vermuten lässt, der CDU nahe steht.
  Die
  "`Liberale Hochschulgruppe"' (LHG) ist eine der FDP nahestehende
  Hochschulgruppe. % , die sich im wesentlichen aus Medizinern und WiWis
%  rekrutiert.
  Besonders aktiv sind die beiden Gruppen allerdings nicht und fallen
meist nur durch mehr oder weniger sinnige Wahlkampfvorschläge auf. Die
  "`JuSo-Hochschulgruppe"'
  steht -- wie sollte es anders sein -- der gleichnamigen
  SPD-Nachwuchsorganisation nahe.
  Augenblicklich am aktivsten sind die "`Grüne Hochschulgruppe"' (GHG),
  die sich v.a. ökologischer Themen annimmt, die Linke Liste (SDS, solid),
  sowie die "`Fachschaften-Vollversammlung"' (FSVV).
  Die FSVV versucht als Zusammenschluss
  aller Fachschaften all diejenigen Aufgaben zu erledigen, die in
  den "`offiziellen"' Gremien nicht behandelt werden können.  Die FSVV ist
  von keiner Partei abhängig und jeder, der vorbeikommt, kann mitreden.
  Hier sollte in Tübingen die eigentliche Studierendenvertretung stattfinden --
  die Fachschaften machen die Arbeit auf Fakultätsebene, die
  Vollversammlung auf Uniebene. Die Schwerpunkte die dabei gesetzt werden
sind selbstverständlich von dem abhängig, was die Fachschaften in die
Vollversammlung tragen -- so könnt auch ihr direkt Einfluss auf die
Unipolitik nehmen.

% Seitdem die grün-rote Koalition ihr Versprechen umgesetzt hat, die verfasste
% Studierendenschaft
% in Baden-Württemberg wieder einzuführen (nachdem 1977 die CDU sie mit der
% Begründung, Studenten
% seien ja alle Terroristen, abgeschafft wurde), werden im Tübingen verschiedene
% Modelle für
% Tübingen diskutiert. In absehbarer Zeit wird es dann auch eine Urwahl geben.
% Wenn die VS
% dann ihre Arbeit aufnimmt, ist diese eine Körperschaft, so dass die Studenten
% etwa
% Finanzautonomie erlangen.


% Leider hat die FSVV, die traditionell links orientiert ist -- mal mehr, mal
% weniger stark -- in den letzten Semestern wenig Kompromissbereitschaft gezeigt,
% so dass sich viele Studenten mit ihr nicht mehr identifizieren können. In der
% Folge wurde die Gruppe "`Liste für Information und Organisation"' gegründet,
% die ebenfalls keiner Partei zuzuordnen ist und ihre Wurzeln vor allem bei den
% WiWis hat.
%
% Jetzt, wo ihr die Namen schon einmal gelesen habt, könnt ihr ja einmal
%   darauf achten, von wem \emph{wann} etwas zu hören ist.  Ganz lustig ist es
%   auch einmal in die Programme reinzuschauen, da findet sich so
%   manche Kuriosität.

Neben den politischen Gruppen, den Fachschaften und den Gremien gibt es
noch zahlreiche andere Möglichkeiten sich an der Universität zu
engagieren und zu betätigen. Zahlreiche Orchester, Lateinamerika-Gruppen,
Gruppierungen ausländischer Studierender und viele mehr sind an
unserer Uni aktiv -- und wenn für euch das Richtige nicht dabei ist,
könnt ihr es ja immer noch ins Leben rufen.
\vspace{2cm}
