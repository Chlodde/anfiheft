% Fakultaet.tex

Die Tübinger Informatik gibt es seit 1990.  Wir waren damit bis jetzt die
  jüngste Fakultät der Universität.
  Die Tübinger Informatik unterscheidet von vielen anderen dadurch, dass -- auch wegen
 der geisteswissenschaftlichen Tradition unserer Uni -- ein
  großer Wert auf Verantwortung gegenüber der Gesellschaft gelegt
  wird.

Vom Wintersemester 2002/03 bis zum Sommersemester 2010 bildete die Informatik
  mit dem Psychologischen Institut die "`Fakultät für Informations- und
  Kognitionswissenschaften"'.  Diese Fakultät bestand aus zwei
  Instituten, dem Psychologischen Institut und dem Wilhelm-Schickard-Institut
  für Informatik. Das Wilhelm-Schickard-Institut ist auf dem Sand
untergebracht, die Psychologie verteilt sich über die Innenstadt, ist aber
vor allem in der alten Frauenklinik zu finden. 

Zum Wintersemester 2010/11 ist die Fakultät in der neuen Mathematisch-
Naturwissenschaftlichen Großfakultät aufgegangen, der die Informatik und
die Psychologie neben Biologie, Chemie, Geowissenschaften, Mathematik,
Pharmazie/Biochemie, und Physik als Fachbereiche angehören. 
Die einzigartige Konstruktion aus Informatik und Psychologie lebt aber
im gemeinsamen Studiengang "`Kognitionswissenschaft"' weiter.

Unser Institut hat inzwischen 18 Arbeitsbereiche, die die Gebiete praktische
Informatik (Datenbanken, Graphisch-Interaktive Systeme, Programmiersprachen
und Übersetzer, Symbolisches Rechnern), theoretische
Informatik (Diskrete Mathematik, Logik und Sprachtheorie, Algorithmik,
Formale Sprachen), technische Informatik
(Technische Informatik, Kommunikationsnetze, Eingebettete Systeme),
Medieninformatik (Medieninformatik, Mensch-Computerinteraktion, Visual Computing, Informationsdienste), Kognition (Kognitive Systeme und Kognitive Modellierung)
und Bioinformatik (Algorithmen der Bioinformatik, Angewandte Bioinformatik, Integrative Transkriptomik) abdecken. 
An der Fakultät ist auch das Zentrum für Bioinformatik angesiedelt (ZBIT).
