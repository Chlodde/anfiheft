\subsection{Wer ist die Fachschaft?}
Die Fachschaft Informatik ist der Zusammenschluss aller Studierender der Informatik, Bioinformatik, Medieninformatik und Medizininformatik. Im Gegensatz zu vielen anderen Fachschaften verzichten wir auf einen Fachschaftsrat oder Fachschaftsvorstand. Stattdessen treffen alle, die sich engagieren oder engagieren möchten einmal die Woche in der Fachschaftssitzung. Hier bereden wir, was die Woche passiert ist und planen die nächsten Wochen. An aktiven Fachschaftlern sind wir ungefähr 20 Studierende, die sich regelmäßig für die Fachschaft engagieren. Hinzu kommen weitere Fachschaftler, welche unregelmäßig bei unseren Aktionen helfen.
\subsection{Was macht die Fachschaft?}
Die Fachschaft übernimmt zahlreiche Aufgaben im Hochschulalltag. Die wichtigsten wollen wir hier kurz nennen. Wenn euch einige dieser Gebiete interessieren, kommt doch mal vorbei und macht mit!

\textbf{Erstsemesterbetreuung}

Jedes Semester strömen wieder einige hundert neue Informatik-Erstsemester an die Uni und haben erstmal keine Ahnung, wie es dort so läuft. Wie ihr vielleicht schon gemerkt habt, versuchen wir euch den Einstieg durch unsere Veranstaltungen, den Brief und dieses Heft einfacher zu machen. Dafür setzen sich Semester für Semester wieder einige Fachschaftler zusammen und überarbeiten unsere Materialien für euch, planen Veranstaltungen und überlegen, was man noch so machen könnte.\\
Falls ihr Lust habt, dafür zu sorgen, dass kommende Erstsemester auch ein (hoffentlich) so cooles Programm, oder ein noch besseres bekommen als du, dann seid ihr hier genau richtig!

\textbf{Partys und Feste}

Wir veranstalten pro Semester ein Clubhausfest und jeden Sommer unser legendäres Sommerfest auf dem Sand. Hierfür muss alles Mögliche organisiert werden: Von der Getränkebestellung über die Gestaltung von Flyern bis zum Bandcasting. \\
Wenn ihr Spaß daran habt, Veranstaltungen zu organisieren, seid ihr hier genau richtig!
\vfill \pagebreak
\textbf{Social Events}

Mehr oder weniger regelmäßig veranstalten wir Spieleabende und LAN-Partys, um den Austausch auch unter den Semestern zu fördern. Hier gilt es vor allem Räume zu reservieren und Werbung zu machen. \\
Ihr zockt gerne oder spielt gerne Gesellschaftsspiele und habt Lust bei der Organisation entsprechender Events zu helfen? Dann kommt doch einfach mal vorbei!

\textbf{Kommunikation}

Um euch die Kommunikation während und nach dem Studium zu erleichtern und euch nicht zu Diensten irgendwelcher Social Media Dienste zu zwingen, betreibt die Fachschaft eine ganze Reihe von Mailinglisten\footnote{\url{https://www.fsi.uni-tuebingen.de/infos/maillists}}. \\
Wenn ihr Lust habt, mitzuhelfen, dass eure Kommilitonen weiter kommunizieren können, dann seid ihr hier genau richtig!

\textbf{Kassensystem}

Da man doch gerne mal was knabbert oder seine Getränkeflasche zu Hause vergessen hat, betreiben wir auf dem Sand für euch ein kleines Kassensystem. Dort könnt ihr zahlreiche Süßigkeiten und Getränke zum quasi Selbstkostenpreis erwerben. 

\textbf{Gremien}

Wir entsenden jährlich zahlreiche Vertreter in verschiedene Gremien der Universität, der Fakultät und des Fachbereichs. Ein Übersicht über die Gremien und unsere derzeitigen Vertreter findet ihr auf unserer Website\footnote{\url{https://www.fsi.uni-tuebingen.de/gremien/studienkommission}}. \\
Wenn ihr euch für Hochschulpolitik und die Arbeit in Gremien interessierst, seid ihr hier genau richtig!

%\textbf{fsi log}
%
%Das fsi log ist die Infozeitschrift der Fachschaft. In ihm berichten wir in (un)regelmäßigen Abständen von unserer Arbeit und von Neuigkeiten am Institut. \\
%Wenn du gerne Texte schreibst, bist du hier genau richtig!
\subsection{Wie kann ich mitmachen?}
Wir sind immer auf der Suche nach motivierten Menschen! Mitmachen kann bei uns jeder, der Lust hat sich für seine Kommilitonen zu engagieren. Wie ihr nun gesehen habt, kann man sich bei uns in den verschiedensten Bereichen engagieren. Wenn euch ein, zwei oder mehrere der Bereiche interessieren, dann kommt doch einfach mal vorbei. Bei uns braucht man kein Vorwissen, alles was ihr wissen müsst, erklären wir euch gerne!

\vfill
\subsection{Andere Fachschaften}
Hin und wieder kann es sinnvoll und notwendig sein auch eine andere Fachschaft zu kontaktieren, wenn es beispielsweise um Veranstaltungen eines anderen Fachbereichs geht (Altklausuren!). Diese beantworten eure Fragen ebenfalls gerne. Die wichtigsten Kontakte haben wir euch hier aufgelistet:

\textbf{Fachschaft Kognitionswissenschaft}\\
E-Mail: \email{kogni-fachschaft@fsi.uni-tuebingen.de}\\
\\
\textbf{Fachschaft Mathematik}\\
E-Mail: \email{fsm@math.uni-tuebingen.de}\\
\\
\textbf{Fachschaft Biologie}\\
E-Mail: \email{fsbio@uni-tuebingen.de}\\
\\
\textbf{Fachschaft Chemie}\\
E-Mail: \email{fschemie@uni-tuebingen.de}\\
\\
\textbf{Fachschaft Physik}\\
E-Mail: \email{fsphysik@uni-tuebingen.de}\\
\\
\textbf{Fachschaft Psychologie}\\
E-Mail: \email{info@fs-psycho.uni-tuebingen.de}\\
\\
\textbf{Fachschaft Medizin}\\
E-Mail: \email{ info@fachschaftmedizin.de}\\
\\
Eine vollständige Liste aller Fachschaften findet ihr auf der Website der FSRVV\footnote{\url{http://www.fsrvv.de/hochschulpolitik/lokal/fachschaften/}}

\vfill
