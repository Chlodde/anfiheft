Wohnen in Tübingen ist ein großes Thema.  Wohnheime werden von verschiedenen Institutionen betrieben: Studentenwerk AdöR, Tübinger Studentenwerk eV, einigen christlichen Vereinigungen und sonstigen Vereinen. 

Private Angebote kann man über \url{https://www.wg-gesucht.de} und \url{https://www.zwischenmiete.de} finden.

Die Wohnungspreise in Tübingen sind allgemein relativ hoch. Hierbei kommt es selbstverständlich auf die Lage und den Vermieter an. Die Wohnheimplätze sind im Vergleich zu vielen privaten "`Wohn-Angeboten"' etwas günstiger.  Euer ganzes Studi-Leben könnt ihr jedoch nicht im selben Wohnheim verbringen, denn üblicherweise setzen die Träger Fristen von 4 bis 6 Semestern.

Bei der Suche nach einem Zimmer oder einer Wohnung bei privaten Vermietern kann die Zimmervermittlung des Studierendenwerks\footnote{\url{https://www.my-stuwe.de/wohnen/privatzimmervermittlung/}} helfen. Darüber hinaus finden sich in den Mittwochs und Samstagsausgaben des Schwäbischen Tagblatts auch immer wieder Zimmerangebote. Alternativ kann man auch selbest eine eigene Anzeige schalten.

In den Mensa Wilhelmstraße und der Mensa Morgenstelle liegt (kostenlos!) das \emph{Dschungelbuch} aus, das u.a. Informationen zu den Wohnheimen enthält. Sehr empfehlenswert!

Kurzfristig eignet sich auch die Jugendherberge in der Gartenstraße 22, Tel: 07071/23002
